\chapter{Понятие математической модели}
\label{cha:definition}

\section{Основные понятия}

Объект --- система состоящая из множества элементов.
Это может быть ракета, рынок ценных бумаг или популяции животных.
В нашем случае это государство.

Модель несет в себе отражение связей между элементами.
Математическая модель --- это математическое представление реальности.
Экономическая модель --- набор уравнений, характеризирующий экономику в целом или отдельно ее отрасль.

Моделирование --- процесс расчета поведения системы на основе граничных условий и заданных связей между элементами системы.

Алгоритм --- логика расчета поведения системы.
Логика может быть основана на разных математических подходах.

\section{Этапы построения математической модели}

Построение математических моделей в экономике является методом для решения задач оптимального упраления.
Экономико-математическая модель отображает некоторые процессы, которые смоделированы с помощью математических теорем и уравнений.

Построение математических моделей состоит из нескольких этапов:
\begin{itemize}
	\item \textbf{Идентификация.}
	Определение основных параметров объекта.
	\item \textbf{Оценка параметров модели.}
	Выбор переменных модели на основе выбранных параметров.
	\item \textbf{Спецификация модели.}
	Определение связей между параметрами.
	Построение уравнений.
	\item \textbf{Моделирование.}
	Проведение моделирования на основе заданных начальных условий.
	\item \textbf{Анализ полученных результатов.}
\end{itemize}

\section{Классификация математических моделей}

Тип математической модели в первую очередь зависит от типа используемых математических средств.
Например:
\begin{itemize}
	\item \textbf{Линейные и нелинейные модели.}
	Модели, в которых связь между зависимой и независимой переменными могут быть линейными или нелинейными (например, линейная регрессия).
	\item \textbf{Дискретные и непрерывные модели.}
	В дискретных моделях изменение значений параметров связано с конкретными моментами времени.
	В непрерывных моделях значения параметров изменяются плавно во времени.
	\item \textbf{Стохастические модели.}
	Стохастические модели прогнозируют явления при неопределенности исходных данных и строятся с помощью методов математической статистики.
	\item \textbf{Оптимизационные модели.}
	Оптимизационная модель позволяет выбрать наилучший вариант из нескольких различных вариантов по любому из признаков.
\end{itemize}
Естественно, что существуют и другие модели, в том числе и смешанные.

Вся теория построения математических моделей основывается на предположениях, которые не всегда являются правдивыми.
Эти предположения помогают существенно упростить описание какого-либо процесса.
Грамотное построение моделей заключается в создании предположений таким образом, чтобы окончательные результаты оказались наиболее независимыми.
Лучшие модели обычно очень просты, но позволяют глубоко описать устройство мира.

В экономике модель математически описывает отдельные экономические явления.
Проще всего думать о моделях как о вымышленных странах, населенных роботами.
Обычно точно оговаривается, каким образом себя ведут роботы, как правило, они обычно максимизируют полезность.
Также задаются ограничения, с которыми сталкиваются роботы в процессе максимизации своей полезности.
Мы будем обобщать результаты такого поведения с помощью некоторых общих правил.
