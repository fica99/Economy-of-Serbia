\chapter{Идентификация параметров}
\label{cha:ident_params}

Для построения математической модели воспользуемся макроэкономическими показатели статистического агенства ООН \cite{unstat}.
Так как модели у нас рассматривают закрытую экономику, то нас интересуют данные в национальной валюте Республики Сербия (RSD).

Существует два вида цен:
\begin{itemize}
	\item \textbf{Текущие.}
	Цены на какую-либо конкретную дату, например на 1 апреля, либо средние за год цены.
	\item \textbf{Постоянные.}
	Цены определенного периода, принимаемые за основу расчета макроэкономических показателей.
	Эти цены не учитывают уровень инфляции.
	На момент написания работы этот период --- 2015 год.
\end{itemize}
Если периоды текущих и постоянных цен совпадают, то и цены тоже соответственно совпадают.

\begin{table}[ht]
	\centering
	\caption{Специальные обозначения для параметров.}
	\begin{tabular}{|r|l|}
	\hline
	\multicolumn{1}{|r|}{Параметр} & \multicolumn{1}{l|}{Описание} \\ \hline
	$t$                          & $\text{год} - 2010$                                            \\
	$Y$                          & ВВП                                                   \\
	$I$                          & размер инвестиций                                     \\
	$C$                          & общие расходы на потребление                          \\
	$L$                          & количество трудящегося населения\footnotemark         \\ \hline
	\end{tabular}
	\label{tab:desig_of_params}
\end{table}
\footnotetext{Трудящееся население включает людей в возрасте 15 лет и старше, которые способны производить товары и услуги в течение определенного периода. Значение включает людей, которые в настоящее время работают, и людей, которые являются безработными, но ищут работу, а также впервые ищущих работу. Однако не все, кто работают, включены. Неоплачиваемые работники, семейные работники и студенты часто не учитываются, а некоторые страны не учитывают военнослужащих. Численность рабочей силы имеет тенденцию меняться в течение года, когда сезонные работники приходят и уходят.}

Таблицы найденных статистических данных для Республики Сербия представлены в приложении \ref{cha:first_app}.

Первым делом посчитаем индексы цен, определим поведение цен в среднем.
Индекс цен представляет собой соотношение макроэкономических показателей данного периода в текущих и постоянных ценах.
Для этого воспользуемся формулой:
\begin{equation*}
	P(X(t)) = \cfrac{X(t)}{X_{const}(t)}
\end{equation*}
где $X(t)$ --- это макроэкономический показатель за определенный период в текущих ценах.
А $X_{const}(t)$ соответственно в постоянных ценах.

Построим график индекса цен для того, чтобы можно было увидеть динамику поведения цен.
\begin{center}
	\captionof{figure}{Индекс цен (RSD)}
	\begin{tikzpicture}
		\begin{axis}[
			xlabel = $t$,
			ylabel = $P(X(t))$,
			table/col sep = semicolon,
			height = 0.3\paperheight,
			width = 0.5\paperwidth,
			xmin = 1993,
			xmax = 2018,
			/pgf/number format/1000 sep={},
			legend pos = south east,
		]
		\legend{Потребление, Инвестиции, Экспорт, Импорт, ВВП};
		\addplot table [x={Year}, y={C}] {tables/Index_prices.csv};
		\addplot table [x={Year}, y={J}] {tables/Index_prices.csv};
		\addplot table [x={Year}, y={E}] {tables/Index_prices.csv};
		\addplot table [x={Year}, y={I}] {tables/Index_prices.csv};
		\addplot table [x={Year}, y={Y}] {tables/Index_prices.csv};
		\end{axis}
	\end{tikzpicture}
\end{center}
В дальнейшем, при построении сложных моделей этот показатель может играть очень важную роль.
