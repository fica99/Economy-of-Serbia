\glava{Анализ стартегий развития в Республике Сербия}

В настоящее время наблюдается явная нехватка эконометрических работ, связанных с построением математических моделей экономики различных государств.
Однако многие рейтинговые агенства регулярно прогнозируют различные варианты развития государства, пишут отчеты, строят стратегии экономичекого роста и т.~д.

\subsection{Всемирный банк}

Согласно отчетам Всемирного банка экономика Сербии может расти быстрее, чем в настоящее время (3 -- 4 процента в год).
В отчете \cite{worldbank_cem} и связанных с ним документах \cite{worldbank_investment,worldbank_financing, worldbank_productivity, worldbank_encouraging, worldbank_labormarket, worldbank_barriers, worldbank_aid, worldbank_workforce} изложена стратегия, которая может помочь экономике страны расти быстрее.
Всемирный банк считает, что текущие темпы роста недостаточно быстро приближают страну к среднему уровню жизни в Европейском Союзе.
Опираясь на новую стратегию Сербия может расти в среднем на 7 процентов в год, удваивая свои доходы за 10 лет.

В стратегии намечены семь ключевых областей действий, которые могли бы сделать возможным этот уровень экономического роста.
Улучшение этих параметров принесло бы наибольшую пользу для роста.
\begin{itemize}
	\item \textbf{Увеличение инвестиций.}
	Увеличение государственных инвестиций и содействие частным инвестициям до уровня поддержит стабильность высоких темпов роста.
	\item \textbf{Финансирование для растущих фирм.}
	Увеличение кредита частному сектору до уровня, близкого к европейским стандартам, расширит финансирование для малых и средних предприятий.
	\item \textbf{Квалифицированные рабочие.}
	Поскольку более двух третей фирм не могут найти работников для осуществления расширения, повышение качества образования может увеличить темпы роста ВВП.
	\item \textbf{Повышение производительности.}
	Повышение производительности труда позволят увеличить производство с добавленной стоимостью, увеличить количество рабочих мест и повысить заработную плату.
	\item \textbf{Содействие экспорту.}
	Сербские экспортеры в среднем в два раза продуктивнее других фирм; увеличение экспорта будет способствовать росту.
	Улучшение инфраструктуры и устранение заграничных ограничений увеличит экспорт.
	\item \textbf{Улучшение правоприменения.}
	Усовершенствованная нормативно-правовая база, включая улучшенную предсказуемость и прозрачность административных процедур, могла бы сократить расходы для бизнеса.
	\item \textbf{Развязывание конкуренции.}
	Сокращение государственного присутствия в экономике уменьшит барьеры для конкуренции, устранит искажения и поможет сэкономить.
\end{itemize}

Ничто из этого не будет легким, но Сербия может сохранить свою с трудом завоеванную макроэкономическую стабильность и вывести свои экономические преобразования на новый уровень.
Таким образом, данная работа представляет интерес, учитывая явную нехватку подобных работ, как для экономики Сербии, так и в научном кругу.
