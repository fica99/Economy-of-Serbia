\glava{Анализ развития экономики в Республике Сербия}

В настоящее время наблюдается явная нехватка эконометрических работ, связанных с построением математических моделей экономики различных государств.
Однако многие рейтинговые агенства регулярно прогнозируют различные варианты развития государства, пишут отчеты, строят стратегии экономичекого роста и т.~д.

\subsection{Последние экономические события}

Рост в 2019 году несколько снизился по сравнению с 2018 годом, но оставался устойчивым на уровне 4,2 процента, что обусловлено увеличением государственных инвестиций наряду с высокими показателями \nom{ПИИ}{Прямые иностранные инвестиции}.

Потребление продолжало быть сильным.
Вклад чистого экспорта в рост был отрицательным, поскольку экспорт рос не так быстро, как в прошлых годах.
Если посмотреть на отраслевой состав, то в 2019 году промышленность увеличилась всего на 0,3 процента, а объем производства в сельском хозяйстве в целом остался таким же, как в 2018 году.
Одна наряду со строительным сектором, услуги внесли значительный вклад в рост ВВП.

Уровень активности и уровень занятости среди населения в возрасте 15 лет и старше в четвертом квартале 2019 года продолжали расти.
Уровень безработицы снизился до 9,7 процента в последнем квартале 2019 года.

Благодаря этим тенденциям уровень бедности снизился с 25,8 процента в 2015 году до примерно 18,9 процента в 2019 году.

К концу 2019 года государственный долг Сербии сократился до 52,9 процента ВВП.
Инфляция была низкой и стабильной.
%При низком инфляционном давлении \nom{НБС}{Народный банк Сербии} понизил свою учетную ставку до 1,75 процента в марте 2020 года.

Приток ПИИ оставался высоким в 2019 году.
Общий объем кредитов вырос на 8,5 процента, в то время как просроченные кредиты сократились до 4,1 процента в декабре 2019 года.

\subsection{Стратегия развития Всемирного банка}

Согласно отчетам Всемирного банка экономика Сербии может расти быстрее, чем в настоящее время (3 -- 4 процента в год).
В отчете \cite{worldbank_cem} и связанных с ним документах \cite{worldbank_investment,worldbank_financing, worldbank_productivity, worldbank_encouraging, worldbank_labormarket, worldbank_barriers, worldbank_aid, worldbank_workforce} изложена стратегия, которая может помочь экономике страны расти быстрее.
Всемирный банк считает, что текущие темпы роста недостаточно быстро приближают страну к среднему уровню жизни в Европейском Союзе.
Опираясь на новую стратегию Сербия может расти в среднем на 7 процентов в год, удваивоив свои доходы за 10 лет.

В стратегии намечены семь ключевых областей действий, которые могли бы сделать возможным этот уровень экономического роста.
Улучшение этих параметров принесло бы наибольшую пользу для роста.
\begin{itemize}
	\item \textbf{Увеличение инвестиций.}
	Увеличение государственных и частным инвестиций поддержит стабильность высоких темпов роста.
	\item \textbf{Финансирование для растущих фирм.}
	Увеличение кредита частному сектору до уровня, близкого к европейским стандартам, расширит финансирование для малых и средних предприятий.
	\item \textbf{Квалифицированные рабочие.}
	Поскольку более двух третей фирм не могут найти работников для осуществления расширения, повышение качества образования может увеличить темпы роста ВВП.
	\item \textbf{Повышение производительности.}
	Повышение производительности труда позволит увеличить производство с добавленной стоимостью, увеличить количество рабочих мест и повысить заработную плату.
	\item \textbf{Содействие экспорту.}
	Сербские экспортеры в среднем в два раза продуктивнее других фирм; увеличение экспорта будет способствовать росту.
	Улучшение инфраструктуры и устранение заграничных ограничений увеличит экспорт.
	\item \textbf{Улучшение правоприменения.}
	Усовершенствованная нормативно-правовая база, включаящая улучшенную предсказуемость и прозрачность административных процедур, могла бы сократить расходы для бизнеса.
	\item \textbf{Развязывание конкуренции.}
	Сокращение государственного присутствия в экономике уменьшит барьеры для конкуренции, устранит искажения и поможет сэкономить.
\end{itemize}

\subsection{Экономический прогноз}

Вспышка пандемии COVID-19 и связанные с ней меры по сдерживанию наносят тяжелый урон мировой экономике, влияют на экономику Сербии и приводят к гораздо более низким темпам роста, чем ожидалось ранее.
Экономика вступит в рецессию в 2020 году, что будет обусловлено снижением туристической и транспортной активности, снижением денежных переводов, замедлением экспорта и сокращением ПИИ и инвестиций в целом. Для смягчения негативных экономических последствий этой пандемии сербские власти обеспечивают всесторонний ответ на кризис.

В среднесрочной перспективе (2021 -- 2023) рост вернется к прежней траектории.
Этот среднесрочный прогноз в решающей степени зависит от международных событий, темпов структурных реформ и политических событий.

Ожидается, что текущие события приведут к небольшому росту бедности в 2020 году.
Помимо непосредственного воздействия на результаты в отношении здоровья, ожидаемое снижение уровня услуг, снижение инвестиций, снижение спроса на сербский экспорт и ограничения мобильности нарушат ситуацию с рабочим местам и трудовыми доходами.
Бедные и уязвимые домохозяйства могут быть затронуты.
Риски связаны, прежде всего, с длительностью и глубиной кризиса, вызванного пандемией COVID-19.
Текущий прогноз предполагает, что меры по сдерживанию могут быть постепенно отменены к концу второго квартала 2020 года.

Ничто из этого не будет легким, но Сербия может сохранить свою с трудом завоеванную макроэкономическую стабильность и вывести свои экономические преобразования на новый уровень.
Таким образом, данная работа представляет интерес, учитывая явную нехватку подобных работ, как для экономики Сербии, так и в научном кругу.
