
\begin{abstract}

	\textbf{Актуальность выбранной темы выпускной квалификационной работы } обусловлена тем, что экономика Республики Сербия в настоящий момент находится на этапе активного развития.
	Математические модели способны описать текущую экономическую ситуацию в стране и спрогнозировать как и положительные, так и отрицательные сюжеты развития государства.

	Экономика государства очень сильно зависит от политической ситуации, она настолько динамична, что построив математическую модель вчера, сегодня она уже может оказаться не актуальной. В связи с вспышкой пандемии COVID-19\footnote{COVID-19 (аббревиатура от англ. COronaVIrus Disease 2019), коронавирусная инфекция 2019-nCoV --- потенциально тяжёлая острая респираторная инфекция, вызываемая коронавирусом \cite{wiki:Coronavirus_disease_2019}.}, многие страны уже оказались в неприятной экономической ситуации.
	Это еще одна причина выбора темы.
	Кроме того, построение математических моделей экономики государства на примере Сербии поможет разобраться как в особенностях государства, так и в математических инструментах.

	\textbf{Объект исследования} --- математические модели экономического роста.

	\textbf{Предмет исследования} --- математическая модель экономики, построенная на примере Республики Сербия.

	\textbf{Цель данной работы} --- смоделировать динамику экономики Республики Сербия в зависимости от поведения внутренних и внешних переменных и сделать выводы.

	Для реализации поставленной цели необходимо решить следующие задачи:
	\begin{enumerate}
		\item Изучить теорию построения математических моделей экономики.
		\item Построить модель на примере макроэкономических данных Республики Сербия.
		\item Сделать выводы.
	\end{enumerate}

	Выпускная квалификационная работа состоит из содержания, перечня сокращений, введения, пяти глав, заключения, списка используемых источников и приложения.

	В первой главе определяются основные термины, описываются этапы построения математических моделей, приводятся различные типы математических моделей.

	Во второй главе описывается теоретическая состовляющая математической модели экономического роста Солоу.

	В третьей главе рассчитываются основные макроэкономические переменные экономики Республики Сербия.

	В четвертой главе проводится анализ, вычисления и построение математической модели экономики Республики Сербия, разбираются полученные результаты.

	В пятой главе исследуются главные сферы экономической деятельности Сербии, оцениваются прогнозы авторитетных рейтинговых агенств (МВФ, Всемирный банк и т.~д.).

\end{abstract}
