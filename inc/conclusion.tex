\Conclusion

В заключение выпускной квалификационной работы можно сделать следующие выводы.

Математичиские модели --- это универсальный инструмент, который в настоящее время становится все более популярным.
Он способен ответить на многие ключевые вопросы.
В дальнейшем знания моделей можно расширять и применять для многих сфер нашей жизни.


В данной выпускной квалификационной работе были изучены:
\begin{enumerate}
	\item Экономическая и политическая ситуация в Республике Сербия.
	\item Производственные функции и их коэффициенты.
	\item Способы построения математических моделей.
	\item Методы поиска оптимальных параметров.
	\item Прогнозы авторитетных рейтинговых агенств.
\end{enumerate}

Кроме того, была построена математическая модель экономического роста Солоу по макроэкономическим данным Республики Сербия и приведены полученные результаты.

Сначала было рассмотрено понятие математической модели и ее основные составляющие.
Затем была описана базовая модель Солоу и ее расширения с учетом научно-технологического прогресса и человеческого капитала.
После этого были описаны основные макроэкономические показатели Республики Сербия и идентифицированны переменные модели.
Затем были описаны все этапы вычислений, были приведены полученные результаты.
В самом конце приведен экономический прогноз Всемирного банка по Республике Сербия, основанный на математичеких моделях.
