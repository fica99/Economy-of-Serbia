\Introduction

Сербия --- индустриально-аграрная страна, расположенная на юго-востоке Европы (центральной части Балканского полуострова).
Через республику проходят важные транспортные и торговые пути, соединяющие Западную и Центральную Европу с Ближним и Средним Востоком.
Сербия располагает значительными сырьевыми ресурсами --- запасами медной, свинцово-цинковой, железной, хромовой, марганцевой руды, а также каменного угля.
В стране имеются значительные гидроэнергетические ресурсы республики (реки Дунай, Морава, Дрина).

На рубеже 1980 -- 1990 годов страна (на тот момент Югославия) была экономически развитой.
Однако политические события 90-х (санкции ООН, война, разрушение инфраструктуры и промышленности в ходе многочисленных воздушных атак НАТО, утрата торговых связей внутри бывшей Югославии и т.~д.) оказали негативное влияние на экономическое и политическое положение страны.

Однако за последние годы в Сербии начался стремительный экономический рост, возобновились иностранная финансовая помощь и инвестиции.
За 10-летний период по данным МВФ сербская экономика выросла почти на 20 процентов.

Сельское хозяйство, промышленность и сектор услуг являются основными источниками доходов Сербии.
Они внесли большой вклад в динамику роста ВВП.
Основная отрасль сельского хозяйства --- растениеводство.
В обрабатывающей промышленности ведущее место занимают машиностроение и металлообработка.
Также уверенными темпами развиваются ИТ и туризм.
Например, экспорт ИТ сектора в 2019 году был выше, чем экспорт доминирующего сельского хозяйства.
Примечательно, что в 2019 году Балканская республика наряду с Ирландией показала самый высокий экономический рост среди всех остальных государств Европы.
В 2020 году Сербия также может показать опережающию темпы роста экономики в Европе.

\textbf{Актуальность выбранной темы выпускной квалификационной работы } обусловлена тем, что экономика Сербии в настоящий момент находится на этапе активного развития.
Математические модели способны описать текущую экономическую ситуацию в стране и спрогнозировать как и положительные, так и отрицательные сюжеты развития государства.

Экономика государства очень сильно зависит от политической ситуации, она настолько динамична, что построив математическую модель вчера, сегодня она уже может оказаться не актуальной. В связи с вспышкой пандемии COVID-19, многие страны уже оказались в неприятной экономической ситуации.
Это еще одна причина выбора темы.
Кроме того, построение математических моделей экономики государства на примере Сербии поможет разобраться как в особенностях государства, так и в математических инструментах.

\textbf{Объект исследования} --- математические модели экономического роста.

\textbf{Предмет исследования} --- математические модели экономики, построенные на примере Республики Сербия.

\textbf{Цель данной работы} --- создание инструмента прогнозирования динамики экономики Республики Сербия в зависимости от поведения внутренних и внешних переменных, сделать выводы и построить прогнозы.

Для реализации поставленной цели необходимо решить следующие задачи:
\begin{enumerate}
	\item Изучить теорию построения математических моделей экономики.
	\item Построить модели на примере макроэкономических данных Республики Сербия.
	\item Сделать выводы и прогнозы.
\end{enumerate}

Выпускная квалификационная работа состоит из содержания, перечня сокращений, введения, пяти глав, заключения, списка используемых источников и приложения.

В первой главе определяются основные термины, описываются этапы построения математических моделей, приводятся различные типы математических моделей.

Во второй главе описывается теоретическая состовляющая математической модели экономического роста Солоу.

В третьей главе рассчитываются основные макроэкономические переменные экономики Республики Сербия.

В четвертой главе проводится анализ, вычисления и построение математической модели экономики Республики Сербия, разбираются полученные результаты.

В пятой главе исследуются главные сферы экономической деятельности Сербии, оцениваются прогнозы авторитетных рейтинговых агенств (МВФ, Всемирный банк и т.~д.).
