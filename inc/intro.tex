\Introduction

Республика Сербия --- индустриально-аграрная страна, расположенная на юго-востоке Европы (Западные Балканы).
Республика является связующим звеном между Центральной Европой и Ближним Востоком.
Через страну проходят ключевые торговые и транспортные пути.
Республика Сербия обладает запасами полезных ископаемых (руды цветных металлов, каменный уголь и т.~д.).
Через страну протекают крупные реки Европы --- Дунай, Сава, Дрина.

На рубеже 1980 -- 1990 годов страна (на тот момент Югославия) была экономически развитой.
Однако трагические события 90-х (война, агрессия НАТО, санкции ООН, развал экономики и т.~д.) оказали негативное влияние на экономическое и политическое положение страны.

За последние годы в Сербии начался стремительный экономический рост, возобновились иностранная финансовая помощь и инвестиции.
За 10-летний период по данным МВФ сербская экономика выросла почти на 20 процентов.

Сельское хозяйство, промышленность и сектор услуг являются основными источниками доходов Сербии.
Они внесли большой вклад в динамику роста ВВП.
Основная отрасль сельского хозяйства --- растениеводство.
В обрабатывающей промышленности ведущее место занимают машиностроение и металлообработка.
Также уверенными темпами развиваются ИТ и туризм.
Например, экспорт ИТ сектора в 2019 году был выше, чем экспорт доминирующего сельского хозяйства.
Примечательно, что в 2019 году Балканская республика наряду с Ирландией показала самый высокий экономический рост среди всех остальных государств Европы.
В 2020 году Сербия также может показать опережающию темпы роста экономики в Европе.
