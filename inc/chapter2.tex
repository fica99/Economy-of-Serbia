\chapter{Модель Солоу}
\label{cha:solow_models}

В этой главе будет описана модель экономического роста, предложенная Робертом Солоу\footnote{Роберт Мертон Солоу --- американский экономист, лауреат Нобелевской премии 1987 года «за фундаментальные исследования в области теории экономического роста»\cite{wiki_solow}.}.
Эта модель способна объяснить, почему одни страны процветают, а другие становятся все беднее.

Предположим, что мир состоит из стран, в которых потребляется только один товар (выпуск).
Из этого следует, что в мире не существует международной торговли.
В модели рассматривается закрытая экономика\footnote{Отсутствие международных сделок.}.
Второе предположение этой модели заключаетя в том, что технологии не зависят от производителей.

\section{Базовая модель Солоу}

Модель основывается на двух уравнениях:
\begin{itemize}
	\item Производственная функция.
	\item Уравнение, описывающее процесс накопления капитала.
\end{itemize}
Производственная функция описывает использование ресурсов для производства выпуска.
Для упрощения модели разделим все ресурсы на труд и капитал.
Введем специальные обозначения, представленные в таблице \ref{tab:prod_func}.
\begin{table}[ht]
	\centering
	\caption{Специальные обозначения для производственной функции}
	\begin{tabular}{|r|l|}
		\hline
		Описание & Обозначение \\ \hline
		Капитал  &      $K$    \\
		Труд     &      $L$    \\
		Выпуск   &      $Y$    \\ \hline
		\end{tabular}%
	\label{tab:prod_func}
\end{table}
Предполагается, что производственная функция является функцией Кобба-Дугласа:
\begin{equation}
	Y = F(K, L) = K^{\alpha}L^{1-\alpha}\text{,}
\label{F:Cob_Duglas}
\end{equation}
где $\alpha$ --- некий коэффициент в интервале от 0 до 1.

Работодатели платят рабочим заработную плату $w$ за каждую единицу труда и платят $r$ за аренду единицы капитала на один период времени.
Пусть цена единицы выпуска равна 1, тогда получим задачу, которая максимизирует прибыль фирме:
\begin{equation}
	\max\limits_{K, L} F(K,L) - rK - wL\text{.}
\end{equation}

Фирмы будут нанимать работников, пока продукт труда не станет равным заработной плате.
\begin{equation*}
	w = \cfrac{\partial F}{\partial L} = \left(1-\alpha\right)\cfrac{Y}{L}\text{.}
\end{equation*}
И арендовать капитал, пока продукт капитала не станет равным его арендной цене.
\begin{equation*}
	r = \cfrac{\partial F}{\partial K} = \alpha\cfrac{Y}{K}\text{.}
\end{equation*}

Заметим, что $wL + rK = Y$, следовательно в сумме платежи за ресурсы соответствуют объему произведенного выпуска, а значит отсутствует прибыль.
Обратим внимание, что доля выпуска, идущего на оплату труда, равна $w\cfrac{L}{Y} = 1 - \alpha$, а на оплату капитала равна $r\cfrac{K}{Y} = \alpha$

Перепишем производственную функцию \ref{F:Cob_Duglas} в расчете на одного трудящегося.
\begin{align*}
y = \cfrac{Y}{L}\text{,}\\
k = \cfrac{K}{L}\text{.}
\end{align*}
Получим производственную функцию выпуска на одного работника:
\begin{equation}
	y = k^{\alpha}\text{.}
\label{F:proiz_per_worker}
\end{equation}

Второе ключевое уравнение модели Солоу описывает процесс накопления капитала:
\begin{equation}
K' = sY-\delta K \text{.}
\label{F:capital_solow}
\end{equation}

Величина в левой части уравнения \ref{F:capital_solow} представляет собой аналог величины $K_{i-1} - K_{i}$ для непрерывного времени.
Это изменение запаса капитала за определенный период.
Второй член выражения обозначает совокупные инвестиции.
Экономика закрытая, поэтому сбережения равны инвестициям.
Третий член уравнения отображает износ капитала в процессе производства.
Часто предполагается, что $\delta = 0,05$.
Это означает, что 5 процентов машин и сооружений выбывает из процесса производвства каждый период.

Рассмотрим темп прироста численности труда $\cfrac{L'}{L}$.
Предположим, что уровень участия в рабочей силе постоянен, а темп припоста численности населения обозначается с помощью параметра $n$.
Таким образом, темп прироста численности занятых $\cfrac{L'}{L}$ равен $n$.
Если $n = 0,01$, это обозначает, что тем прироста численности занятых увеличилась на 1 процент за определенный период.
Этот рост можно записать в следующем виде:
\begin{equation*}
	L(t) = L_{0}e^{nt}\text{.}
\end{equation*}

Для того, чтобы показать каким образом во времени меняется выпуск в расчете на одного работника, возьмем логарифм, а затем производные по времени.
Например:
\begin{equation*}
	k = \cfrac{K}{L}\Rightarrow \log{k}=\log{K} - log{L} \Rightarrow \cfrac{k'}{k} = \cfrac{K'}{K} - \cfrac{L'}{L}\text{.}
\end{equation*}
Произведя такую же операцию для уравнения \ref{F:capital_solow}, получим:
\begin{equation*}
\cfrac{k'}{k} = \cfrac{sY}{K} - n - \delta = \cfrac{sy}{k} - n - \delta\text{.}
\end{equation*}
Это преобразование позволило получить уравнение, показывающее изменение во времени капитала на каждого работника:
\begin{equation*}
k'=sy-(n + \delta)k\text{.}
\end{equation*}

\section{Модель Солоу и научно-технологический прогресс}

Усовершенствуем нашу модель с помощью добавления в производственную функцию переменной, учитывающей научно-технологический прогресс.
Новая производственная функция будет выглядеть следующим образом:
\begin{equation}
	Y=F(K,AL)=K^{\alpha}(AL)^{1 - \alpha}\text{.}
\end{equation}\label{F:Cob_dogl_tech}
Технологический прогресс наблюдается, когда $A$ увеличивается с течением времени --- например, единица труда становится более продуктивной, когда уровень технологий растет.

Важным предположением данной модели является то, что технологический прогресс экзогенен\footnote{Определяется вне модели и могут изменяться со временем.}.
Вместо того, чтобы самостоятельно моделировать откуда технологии появляются, мы упростим этот момент и сделаем предположение, что $A$ растет с постоянным темпом:
\begin{equation*}
	\cfrac{A'}{A} = g \Rightarrow A = A_{0} e^{gt}\text{,}
\end{equation*}
где $g$ это параметр, отвечающий за технологический рост.

Уравнение накопления капитала остается прежним. Перепишем его в новом виде:
\begin{equation}
\cfrac{K'}{K} = s \cfrac{Y}{K} - \delta\text{.}
\end{equation}\label{F:capital_solow_tech}
Перепишем произведенную функцию \ref{F:Cob_dogl_tech} в расчете на одного рабочего:
\begin{equation*}
y=k^{\alpha}A^{1 - \alpha}\text{.}
\end{equation*}
Проведем операцию из предыдущего параграфа.
Возьмем логарифм и продифференцируем:
\begin{equation}
\cfrac{y'}{y} = \alpha \cfrac{k'}{k} + (1 - \alpha) \cfrac{A'}{A}\text{.}
\label{F:cob_dogl_per_person_tech}
\end{equation}
Заметим из уравнения \label{F:capital_solow_tech}, что рост капитала $K$ будет постоянным только, если $\cfrac{Y}{K}$ константа.
Если $\cfrac{Y}{K}$ постоянно, то $\cfrac{y}{k}$ тоже постоянно, а значит $y$ и $k$ будут расти с одинаковым темпом.
Если капитал, выпуск, потребление и население растут с постоянными темпами, то эта ситуация называется траекторией сбалансированного роста.

Введем обозначение $g_{x}$ --- уровень роста переменной $x$ по траектории сбалансированного роста.
Тогда получаем, что $g_{k} = g_{y}$. Подставляя это равенство и $\cfrac{A'}{A}$ в уравнение \ref{F:cob_dogl_per_person_tech} получаем:
\begin{equation*}
g_{y} = g_{k}=g\text{.}
\end{equation*}
Из этого следует, что производительность и капитал на одного работника растут со скоростью технологических изменений.
В предыдущей главе не было технологического роста, а значит и роста производительности работников не было.
Из этого следует, что $g_{y}=g_{k}=g=0$.

Модель с учетом технологий показывает, что научно-технический прогресс является источником устойчивого роста на душу населения.

\section{Модель Солоу с человеческим капиталом}

Заметим, что модель может быть улучшена, включив в нее человеческий капитал, то есть труд в разных экономиках может иметь разные уровни образования и навыки.

Предположим, что выпуск $Y$ получается путем объединения капитала $K$, с квалифицированным трудом $H$.
Получим новую производственную функцию:
\begin{equation}
Y=K^{\alpha}(AH)^{1-\alpha}\text{,}
\end{equation}
где $A$ отвечает за научно-технологический прогресс с уровнем роста $g$.

Люди в этой модели накапливают человеческий капитал, тратя время на обучение.
Обозначим $u$ за долю времени, потраченного на обучение навыкам, а $L$ --- общее количество необученного труда, используемого в производстве.
Предположим, что обучение неквалифицированного труда професиональным навыкам за время $u$ производит опытный труд $H$ в соответствии с:
\begin{equation}
H=e^{\psi u} L \text{,}
\label{F:labor_skill}
\end{equation}
где $\psi$ является положительной константой.
Заметим, что если $u = 0$, тогда $H = L$, весь труд --- неквалифицированный.
Увеличивая $u$, единица неквалифицированного труда увеличивает эффективные единицы опытного труда $H$.
Для того, чтобы посчитать это изменение, возьмем логарифм и посчитаем производную от уравнения \ref{F:labor_skill}:
\begin{equation*}
\cfrac{d \log{H}}{du} = \psi \Rightarrow \cfrac{dH}{du}=\psi H\text{.}
\end{equation*}
Для разъяснения этого равенства предположим, что $u$ увеличивается на единицу (например, добавление еще одного года обучения в школе) и $\psi = 0.10$.
В этом случае $H$ увеличивается на 10 процентов.
Как было показано ранее, физический капитал накапливается путем инвестирования некоторого объема продукции:
\begin{equation*}
K'=s_{K}Y - \delta K \text{,}
\end{equation*}
где $s_{K}$ это инвестиционная ставка для физического капитала и $\delta$ --- константная норма амортизации.

Теперь перепишем производственную функцию с точки зрения выпуска на одного работника:
\begin{equation}
y = k^{\alpha}\left(Ah\right)^{1 - \alpha}\text{,}
\label{F:proizv_per_person}
\end{equation}
где $h=e^{\psi u}$
Мы предполагаем, что отдельные лица экономят и инвестируют постоянную часть своего дохода, а значит что $u$ постоянна и дана экзогенно.

Константа $h$ показывает, что производственная функция \ref{F:proizv_per_person} очень похожа на ту, которая использовалась ранее.
В частности, $y$ и $k$ будут расти с постоянной скоростью технологического прогресса $g$ .

Рассмотрим переменные, которые являются постоянными на сбалансированной траектории роста.
Обозначим эти переменные с тильдой.
Так как $h$ константа можем определить эти переменные делением функции \ref{F:proizv_per_person} на $Ah$:
\begin{equation}
\tilde{y} = \tilde{k}^{\alpha}\text{,}
\end{equation}
оно эквивалентно уравнению \ref{F:proiz_per_worker}.

Запишем уравнение накопления капитала в этом же виде:
\begin{equation*}
\tilde{k}' = s_{K} \tilde{y} - (n + g + \delta) \tilde{k}\text{.}
\end{equation*}
Заметим, что модель идентична модели, которая рассматривалась ранее.
Это означает, что все результаты, которые обсуждались ранее относительно динамики модели Солоу, применимы здесь.
Добавление человеческого капитала, не меняет базовую модель.

Таким образом, страны богаты, потому что имеют высокий уровень инвестиций в физический капитал, тратят большую часть времени на накопление навыков.
