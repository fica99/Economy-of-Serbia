\chapter{Модель Солоу}
\label{cha:solow_models}

В этой главе будет описана модель экономического роста, предложенная Робертом Солоу\footnote{Роберт Мертон Солоу --- американский экономист, лауреат Нобелевской премии 1987 года «за фундаментальные исследования в области теории экономического роста»\cite{wiki_solow}.}.
Эта модель способна объяснить, почему одни страны процветают, а другие становятся все беднее.

Предположим, что мир состоит из стран, в которых потребляется только один товар (выпуск).
Из этого следует, что в мире не существует международной торговли.
В модели рассматривается закрытая экономика\footnote{Отсутствие международных сделок.}.
Второе предположение этой модели заключаетя в том, что технологии не зависят от производителей.

\section{Базовая модель Солоу}

Модель основывается на двух уравнениях:
\begin{itemize}
	\item Производственная функция.
	\item Уравнение, описывающее процесс накопления капитала.
\end{itemize}
Производственная функция описывает использование ресурсов для производства выпуска\footnote{Выпуск состоит не только из физических единиц.}.
Для упрощения модели разделим все ресурсы на труд и капитал.
Введем специальные обозначения, представленные в таблице \ref{tab:prod_func}.
\begin{table}[ht]
	\centering
	\caption{Специальные обозначения для производственной функции}
	\begin{tabular}{|r|l|}
		\hline
		Описание & Обозначение \\ \hline
		Капитал  &      $K$    \\
		Труд     &      $L$    \\
		Выпуск   &      $Y$    \\ \hline
		\end{tabular}%
	\label{tab:prod_func}
\end{table}
Предполагается, что производственная функция является функцией Кобба-Дугласа:
\begin{equation}
	Y = F(K, L) = K^{\alpha}L^{1-\alpha}\text{,}
\label{F:Cob_Duglas}
\end{equation}
где $\alpha$ --- некий коэффициент в интервале от 0 до 1.

Работадатель платят рабочим заработную плату $w$ за каждую единицу труда и платят $r$ за аренду единицы капитала на один период времени.
Пусть цена единицы выпуска равна 1, тогда получим задачу, которая максимизирует прибыть фирме:
\begin{equation}
	\max\limits_{K, L} F(K,L) - rK - wL\text{.}
\end{equation}

Фирмы будут нанимать работников, пока продукт труда не станет равным заработной плате.
\begin{equation*}
	w = \cfrac{\partial F}{\partial L} = \left(1-\alpha\right)\cfrac{Y}{L}\text{.}
\end{equation*}
И арендовать капитал, пока продукт капитала не станет равным его арендной цене.
\begin{equation*}
	r = \cfrac{\partial F}{\partial K} = \alpha\cfrac{Y}{K}\text{.}
\end{equation*}

Заметим, что $wL + rK = Y$, следовательно в сумме платежи за ресурсы соответствуют объему произведенного выпуска, а значит отсутствует прибыль.
Обратим внимание, что доля выпуска, идущего на оплату труда, равна $w\cfrac{L}{Y} = 1 - \alpha$, а на оплату капитала равна $r\cfrac{K}{Y} = \alpha$

Перепишем производственную функцию \ref{F:Cob_Duglas}в расчете на одного трудящегося.
\begin{align*}
y = \cfrac{Y}{L}\text{,}\\
k = \cfrac{K}{L}\text{.}
\end{align*}
Получим производственную функцию выпуска на одного работника:
\begin{equation*}
	y = k^{\alpha}\text{.}
\end{equation*}

Второе ключевое уравнение модели Солоу описывает процесс накопления капитала:
\begin{equation}
\dot{K} = sY-\delta K \text{.}
\label{F:capital_solow}
\end{equation}

Величина в левой части уравнения \ref{F:capital_solow} представляет собой аналог величины $K_{i-1} - K_{i}$ для непрерывного времени.
Это изменение запаса капитала за определенный период.
Второй член выражения обозначает совокупные инвестиции.
Экономика закрытая, поэтому сбережения равны инвестициям.
Третий член уравнения отображает износ капитала в процессе производства.
Часто предполагается, что $\delta = 0,05$.
Это означает, что 5 процентов машин и сооружений выбывает из процесса производвства каждый период.

Рассмотрим темп прироста численности труда $\cfrac{\dot{L}}{L}$.
Предположим, что уровень участия в рабочей силе постоянен, а темп припоста численности населения обозначается с помощью параметра $n$.
Таким образом, темп прироста численности занятых $\cfrac{\dot{L}}{L}$ равен $n$.
Если $n = 0,01$, это обозначает, что тем прироста численности занятых увеличилась на 1 процент за определенный период.
Этот рост можно записать в следующем виде:
\begin{equation*}
	L(t) = L_{0}e^{nt}\text{.}
\end{equation*}

Для того, чтобы показать каким образом во времени меняется выпуск в расчете на одного работника, возьмем логарифм, а затем производные по времени.
Например:
\begin{equation*}
	k = \cfrac{K}{L}\Rightarrow \log{k}=\log{K} - log{L} \Rightarrow \cfrac{\dot{k}}{k} = \cfrac{\dot{K}}{K} - \cfrac{\dot{L}}{L}\text{.}
\end{equation*}
Произведя такую же операцию для уравнения \ref{F:capital_solow}, получим:
\begin{equation*}
\cfrac{\dot{k}}{k} = \cfrac{sY}{K} - n - \delta = \cfrac{sy}{k} - n - \delta\text{.}
\end{equation*}
Это преобразование позволило получить уравнение, показывающее изменение во времени капитала на каждого работника:
\begin{equation*}
\dot{k}=sy-(n + \delta)k\text{.}
\end{equation*}

\section{Модель Солоу и научно-технический прогресс}
