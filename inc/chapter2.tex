\chapter{Модель Солоу}
\label{cha:solow_models}

В этой главе будет описана модель экономического роста, предложенная Робертом Солоу\footnote{Роберт Мертон Солоу --- американский экономист, лауреат Нобелевской премии 1987 года «за фундаментальные исследования в области теории экономического роста»\cite{wiki_solow}.}.
Эта модель способна объяснить, почему одни страны процветают, а другие становятся беднее.

Предположим, что мир состоит из стран, в которых потребляется только один товар (выпуск) и не существует международной торговли.
В модели рассматривается закрытая экономика\footnote{Отсутствие международных сделок.}.
Второе предположение этой модели заключаетя в том, что технологии не зависят от производителей.

\section{Базовая модель Солоу}

Модель основывается на двух уравнениях:
\begin{itemize}
	\item Производственная функция.
	\item Уравнение, описывающее процесс накопления капитала.
\end{itemize}

Для упрощения модели разделим все ресурсы на капитал и труд.
Фирма выпускает товар , используя капитал и труд.
Сделаем предположение, что выпуск нельзя отложить на время.
Он может быть потреблен или повторно инвестирован (в этом же периоде), но не может просто храниться.

Труд --- это единица, показывающая сколько времени люди тратят на работу.
Если труд не используется в течение определенного периода, то он пропадает навсегда.

Капитал выражается в единицах товаров.
Капитал отличается от труда тем, что он должен производиться (труд --- это дар).
Также запас капитала не исчерпывается в течение времени (использование капитала сегодня не мешает вам использовать его для производства завтра).

Для понимания можно провести аналогию с фруктами.
Капитал --- фруктовое дерево, которое может быть посажено, а семя --- несъеденный фрукт.
Фруктовое дерево может существовать самостоятельно и может давать плоды в течение нескольких периодов.
Производственный процесс включает деревья (капитал), которые приносят фрукты, и людей(труд), которые тратят время на сбор плодов.

Введем специальные обозначения, представленные в таблице \ref{tab:prod_func}.
\begin{table}[ht]
	\centering
	\caption{Специальные обозначения для производственной функции.}
	\begin{tabular}{|r|l|}
		\hline
		Описание & Обозначение \\ \hline
		Капитал  &      $K$    \\
		Труд     &      $L$    \\
		Выпуск   &      $Y$    \\ \hline
		\end{tabular}%
	\label{tab:prod_func}
\end{table}
Мы предполагаем, что существует некая функция, объединяющая капитал и труд, для производства выпуска.
Такая функция называется производственной.

Эта функция обладает следующими свойствами:
\begin{enumerate}
	\item И труд и капитал необходимы для производства.
	\[
		F(K, 0) = F(0, L) = 0
	\]
	\item Если удвоить труд и капитал, то удвоится выпуск.
	\[
		F(\gamma K, \gamma L) = \gamma F(K, L), \gamma > 0
	\]
	\item При одном фиксированном аргументе увеличение второго аргумента увеличивает выпуск.
\end{enumerate}
Это означает, что функция является возрастающей и вогнутой.

Функцией, удовлетворяющей этим свойствам, является производственная функция Кобба-Дугласа:
\begin{equation}
	Y = F(K, L) = K^{\alpha}L^{1-\alpha}\text{,}
\label{F:Cob_Duglas}
\end{equation}
где $\alpha$ --- некий коэффициент в интервале от 0 до 1.

Говорят, что,
\begin{itemize}
\item если $\forall \alpha>0$ $F(\alpha K, \alpha L) = \alpha Y$, то функция демонструрует постоянную отдачу от масштаба.
\item если $F(\alpha K, \alpha L) > \alpha Y$, то производственная функция показывает возврастающую отдачу от масштаба.
\item если $F(\alpha K, \alpha L) < \alpha Y$, то убывающую отдачу от масштаба.
\end{itemize}
Основываясь на свойстах производтвенной функции, можно сделать вывод, что она демонструрует постоянную отдачу от масштаба.


Пусть фирма платит рабочим заработную плату $w$ за каждую единицу труда и платит $r$ за аренду единицы капитала на один период времени.
Посчитаем доход фирмы:
\begin{equation*}
	\Pi=F(K, L) - rK - wL\text{.}
\end{equation*}

Фирма хочет максимизировать свою прибыль, которая равна выручке с вычетом затрат.
Теперь получаем задачу, которая максимизирует прибыль фирме:
\begin{equation}
	\max\limits_{K, L} F(K,L) - rK - wL\text{.}
\end{equation}

Решение данной задачи заключается во взятии частных производных по каждой переменной и приравнивании к нулю.
\begin{align*}
	\cfrac{\partial \Pi}{\partial K} &= 0\text{,} \\
	\cfrac{\partial \Pi}{\partial L} &= 0\text{.}
\end{align*}
Посчитав производные получим:
\begin{align*}
	\alpha K^{\alpha - 1} L^{1-\alpha} &= r\text{,}\\
	(1 - \alpha) K^{\alpha} L^{-\alpha} &= w\text{.}\\
\end{align*}

Заметим, что $wL = (1-\alpha)Y$, $rK = \alpha Y$, а значит $\Pi=Y - \alpha Y - (1\hm{-} \alpha)Y = 0$.
Следовательно в сумме платежи за ресурсы соответствуют объему произведенного выпуска, а значит отсутствует прибыль.
Этот результат является свойством производственной функции с постоянной отдачей от масштаба.
В реальном мире имеется множество внешних факторов, влияющих на прибыль.

Обратим внимание, что доля выпуска, идущего на оплату труда, равна $w\cfrac{L}{Y} = 1 - \alpha$, а на оплату капитала равна $r\cfrac{K}{Y} = \alpha$

Перепишем производственную функцию \ref{F:Cob_Duglas}, зависящую от капитала, в расчете на одного трудящегося.
Иногда этот показатель является очень значимым --- во многих авторитетных источниках приведена статистика на душу населения.
\begin{align*}
y = \cfrac{Y}{L}\text{,}\\
k = \cfrac{K}{L}\text{.}
\end{align*}
Получим производственную функцию выпуска на одного работника:
\begin{equation}
	y = k^{\alpha}\text{.}
\label{F:proiz_per_worker}
\end{equation}

Нарисуем график.
Чем выше уровень капитала на трудящегося, тем больше его выпуск.

\begin{center}
\begin{tikzpicture}
	\begin{axis}[
		xlabel = {$k$},
		ylabel = {$y$},
		ticks = none
	]
	\addplot[blue] {x^0.1};
	\end{axis}
\end{tikzpicture}
\captionof{figure}{График производственной функции на душу населения}
\end{center}
Однако для капитала на одного работника характерна убывающая предельная отдача, каждая дополнительная единица капитала, полученная рабочим, будет увеличивать его выпуск на все меньшую величину.

Сделаем важные предположения:
\begin{itemize}
	\item Пусть люди каждый период откладывают долю выпуска на будущее увеличение капитала (инвестиции), а оставшуюся доля потребляют на личные расходы.
	\item Каждый период существующий капитал  изнашивается с постоянной долей.
\end{itemize}

Теперь можно записать второе ключевое уравнение модели Солоу, которое описывает процесс накопления капитала:
\begin{equation}
K' = sY-\delta K \text{,}
\label{F:capital_solow}
\end{equation}
где $K' = \cfrac{dK}{dt}$ --- производная по времени.

В соответствии с ним изменение запаса капитала $K'$ за определенный период равно совокупным инвестициям $sY$ с вычетом износа капитала в процессе производства $\delta K$.
Величина в левой части уравнения представляет собой аналог дискретной величины $K_{i+1} - K_{i}$.

Второй член выражения \ref {F:capital_solow} представляет собой совокупные инвестиции.
Из предположения следует, что люди сохраняют постоянную часть $s$ своего дохода: заработной и арендной платы $Y=wL+rK$.
Экономика закрытая, поэтому сбережения равны инвестициям, а значит $I=sY$.
По предположению выше получаем, что люди потребляют фиксированную долю дохода $C=(1-s)Y$.

Третий член уравнения \ref {F:capital_solow} отражает износ капитала в процессе производства.
Предполагается, что постоянная часть капитала изнашивается каждый период (вне зависимости от производства).
Часто предполагается, что $\delta = 0.05$.
Это означает, что 5 процентов машин и сооружений выбывает из процесса производвства каждый период.

Рассмотрим темп прироста численности труда $\cfrac{L'}{L}$.
Предположим, что уровень участия в рабочей силе постоянен, а темп припоста численности населения обозначается с помощью параметра $n$.
Таким образом, темп прироста численности занятых $\cfrac{L'}{L}$ равен $n$.
Если $n = 0.01$, это обозначает, что тем прироста численности занятых увеличилась на 1 процент за определенный период.
Этот экспоненциальный рост можно записать в следующем виде:
\begin{equation*}
	L(t) = L_{0}e^{nt}\text{.}
\end{equation*}

Воспользуемся математическим приемом, возьмем логарифм, а затем производные по времени.
Например:
\begin{equation*}
	k = \cfrac{K}{L}\Rightarrow \log{k}=\log{K} - log{L} \Rightarrow \cfrac{k'}{k} = \cfrac{K'}{K} - \cfrac{L'}{L}\text{.}
\end{equation*}

Для того, чтобы показать каким образом во времени меняется выпуск в расчете на одного работника произведем эту операцию от уравнения накопления капитала \ref{F:capital_solow}.
Получим:
\begin{equation*}
\cfrac{k'}{k} = \cfrac{sY}{K} - n - \delta = \cfrac{sy}{k} - n - \delta\text{.}
\end{equation*}
Это преобразование позволило получить уравнение, показывающее изменение во времени капитала на каждого работника:
\begin{equation*}
k'=sy-(n + \delta)k\text{.}
\end{equation*}

\section{Модель Солоу и научно-технологический прогресс}

Усовершенствуем нашу модель с помощью добавления в производственную функцию переменной, учитывающей научно-технологический прогресс.
Новая производственная функция будет выглядеть следующим образом:
\begin{equation}
	Y=F(K,AL)=K^{\alpha}(AL)^{1 - \alpha}\text{.}
\label{F:Cob_dogl_tech}
\end{equation}
Технологический прогресс наблюдается, когда $A$ увеличивается с течением времени --- например, единица труда становится более продуктивной, когда уровень технологий растет.

Важным предположением данной модели является то, что технологический прогресс экзогенен\footnote{Определяется вне модели и могут изменяться со временем.}.
Вместо того, чтобы самостоятельно моделировать откуда технологии появляются, мы упростим этот момент и сделаем предположение, что $A$ растет с постоянным темпом:
\begin{equation*}
	\cfrac{A'}{A} = g \Rightarrow A_{t} = A_{0} e^{gt}\text{,}
\end{equation*}
где $g$ --- это параметр, отвечающий за технологический рост, а $A_0$ --- начальный уровень технологического прогресса.

Уравнение накопления капитала остается прежним. Перепишем его в новом виде:
\begin{equation}
\cfrac{K'}{K} = s \cfrac{Y}{K} - \delta\text{.}
\label{F:capital_solow_tech}
\end{equation}
Перепишем производственную функцию \ref{F:Cob_dogl_tech} в расчете на одного рабочего:
\begin{equation*}
y=k^{\alpha}A^{1 - \alpha}\text{.}
\end{equation*}
Проведем операцию из предыдущего параграфа.
Возьмем логарифм и продифференцируем:
\begin{equation}
\cfrac{y'}{y} = \alpha \cfrac{k'}{k} + (1 - \alpha) \cfrac{A'}{A}\text{.}
\label{F:cob_dogl_per_person_tech}
\end{equation}
Заметим из уравнения \label{F:capital_solow_tech}, что рост капитала $K$ будет постоянным только, если $\cfrac{Y}{K}$ константа.
Если $\cfrac{Y}{K}$ постоянно, то $\cfrac{y}{k}$ тоже постоянно, а значит $y$ и $k$ будут расти с одинаковым темпом.
Если капитал, выпуск, потребление и население растут с постоянными темпами, то эта ситуация называется траекторией сбалансированного роста.

Введем обозначение $g_{x}$ --- уровень роста переменной $x$ по траектории сбалансированного роста.
Тогда получаем, что $g_{k} = g_{y}$. Подставляя это равенство и $\cfrac{A'}{A}$ в уравнение \ref{F:cob_dogl_per_person_tech} получаем:
\begin{equation*}
g_{y} = g_{k}=g\text{.}
\end{equation*}
Из этого следует, что производительность и капитал на одного работника растут со скоростью технологических изменений.
В предыдущей главе не было технологического роста, а значит и роста производительности работников не было.
Из этого следует, что $g_{y}=g_{k}=g=0$.

Модель с учетом технологий показывает, что научно-технический прогресс является источником устойчивого роста на душу населения.

\section{Модель Солоу с человеческим капиталом}

Заметим, что модель может быть улучшена, включив в нее человеческий капитал, то есть труд в разных экономиках может иметь разные уровни образования и навыки.

Предположим, что выпуск $Y$ получается путем объединения капитала $K$, с квалифицированным трудом $H$.
Получим новую производственную функцию:
\begin{equation}
Y=K^{\alpha}(AH)^{1-\alpha}\text{,}
\end{equation}
где $A$ отвечает за научно-технологический прогресс с уровнем роста $g$.

Люди в этой модели накапливают человеческий капитал, тратя время на обучение.
Обозначим $u$ за долю времени, потраченного на обучение навыкам, а $L$ --- общее количество необученного труда, используемого в производстве.
Предположим, что обучение неквалифицированного труда професиональным навыкам за время $u$ производит опытный труд $H$ в соответствии с:
\begin{equation}
H=e^{\psi u} L \text{,}
\label{F:labor_skill}
\end{equation}
где $\psi$ является положительной константой.
Заметим, что если $u = 0$, тогда $H = L$, весь труд --- неквалифицированный.
Увеличивая $u$, единица неквалифицированного труда увеличивает эффективные единицы опытного труда $H$.
Для того, чтобы посчитать это изменение, возьмем логарифм и посчитаем производную от уравнения \ref{F:labor_skill}:
\begin{equation*}
\cfrac{d \log{H}}{du} = \psi \Rightarrow \cfrac{dH}{du}=\psi H\text{.}
\end{equation*}
Для разъяснения этого равенства предположим, что $u$ увеличивается на единицу (например, добавление еще одного года обучения в школе) и $\psi = 0.10$.
В этом случае $H$ увеличивается на 10 процентов.
Как было показано ранее, физический капитал накапливается путем инвестирования некоторого объема продукции:
\begin{equation*}
K'=sY - \delta K \text{,}
\end{equation*}
где $s$ --- это инвестиционная ставка для физического капитала и $\delta$ --- константная норма амортизации.

Теперь перепишем производственную функцию с точки зрения выпуска на одного работника:
\begin{equation}
y = k^{\alpha}\left(Ah\right)^{1 - \alpha}\text{,}
\label{F:proizv_per_person}
\end{equation}
где $h=e^{\psi u}$.
Как рассчитать сколько времени потратить на учебу, а сколько работать?
Так же как мы предполагаем, что отдельные лица экономят и инвестируют постоянную часть своего дохода, предположим, что $u$ постоянна и задана вне модели.

Факт того, что $h$ --- константа, означает, что производственная функция \ref{F:proizv_per_person} очень похожа на ту, которая использовалась ранее.
В частности, $y$ и $k$ будут расти с постоянной скоростью технологического прогресса $g$ .
Заметим, что модель идентична модели, которая рассматривалась ранее.
Это означает, что все результаты, которые обсуждались ранее относительно динамики модели Солоу, применимы здесь.
Добавление человеческого капитала, не меняет базовую модель.

Таким образом, можно сделать предположение, что страны богаты, потому что имеют высокий уровень инвестиций в физический капитал и тратят большую часть времени на накопление навыков.
