\chapter{Модель Республики Сербия}

Мы разобрались с основами модели Солоу, теперь надо ее построить.
В общем случае модель состоит из нескольких уравнений, которые описывают взаимосвязи между набором эндогенных переменных --- переменных, значения которых определяются внутри модели.
Можно заметить, что уравнения, описывающие взаимосвязи между эндогенными переменными включают в себя различные параметры и экзогенные переменные.
Параметры --- это постоянные величины.
Экзогенные переменные --- это величины, которые могут меняться во времени, однако их значения определяются вне модели.

\section{Определение уравнений}

После объяснения этих понятий мы готовы построить модель.
Решение модели означает получение значений каждой эндогенной переменной, когда даны значения для экзогенных переменных и параметров.
В идеале хотелось бы иметь возможность выражать каждую эндогенную переменную как функцию только от экзогенных переменных и параметров.

В предыдущем разделе мы ввели два ключевых уравнения модели Солоу:
\begin{align*}
	Y=K^{\alpha}(AH)^{1-\alpha}\text{,}\\
	K'=s_{K}Y - \delta K \text{.}
\end{align*}

Переопределим обозначение $t$, чтобы было удобнее работать.
Период времени будет один год.
Пусть 2010 год будет базисным годом, тогда $t = \text{год} - 2010$.

Для начала рассчитаем труд, пусть $L_0 = L_{stat}(0)$.
Теперь воспользуемся формулой рассчета количества труда:
\begin{equation*}
	L(t) = L_0 e^{nt}\text{,}
\end{equation*}
где $n$ --- темп прироста численности трудящегося населения.

Теперь, воспользовавшись количеством трудящихся, можно вычислить квалифицированный труд.
Cделаем предположение, что только половина трудящихся страны является квалифицированной.
Из-за отсутствия статистики приходится делать предположения.
В зависимости от уровня образования, обучение занимает разное время.
В среднем квалифицированное обучение в Республике Сербия занимает 4 года, значит $u = 4$.
Модифицировав формулу подсчета квалифицированного труда в соответствии с предположениями, получим формулу:
\begin{equation*}
H(t) = 0.5L(t) + 0.5 e^{\psi u}L(t)\text{,}
\end{equation*}
где $\psi$ --- константа, показывающая увеличение эффективности труда за единицу времени обучения.

Теперь займемся технологиями.
Так как уровень технологического прогресса существенно влияет на производство, нам необходимо его определить.
\begin{equation*}
A(t) = A_{0}e^{gt}\text{,}
\end{equation*}
где $g$ --- это параметр, отвечающий за технологический рост.
Начальный уровень технологий $A_0$ будет определен в дальнейшем.

Как было показано ранее, физический капитал накапливается путем инвестирования некоторого объема продукции, а значит можно посчитать начальный запас капитала, как сумма инвестиций c учетом амортизации за прошлые года:
\begin{equation*}
	K_0= \varkappa + \sum\limits_{t = -17}^{0}J_{stat}(t) e^{\delta t}\text{,}
\end{equation*}
где $\varkappa$ --- некая константа для приближения результатов к статистике, $\delta$ --- параметр износа капитала.

Теперь мы можем рассчитать капитал по формуле:
\begin{equation*}
	K'=s_{K}Y - \delta K \text{,}
\end{equation*}
где $s_{K}$ --- норма инвестирования.
Преобразуем это уравнение к более удобному виду для вычислений:
\begin{equation*}
	K_{t+1} - K_{t} = s_{K}Y_t - \delta K_{t}\text{.}
\end{equation*}
Перенесем $K_{t}$ в правую часть уравения:
\begin{equation*}
K_{t+1} = s_{K} Y_t + (1 - \delta)K_{t}\text{.}
\end{equation*}
Теперь мы можем подставить производственную функцию и получим:
\begin{equation*}
	K_{t+1} = s_{K}K_{t}^{\alpha}(AH)^{1-\alpha} + (1 - \delta)K_{t}\text{.}
\end{equation*}
В этой формуле все переменные определены, поэтому без проблем можно рассчитать капитал.

Переменные производственной функции нам известны.
Посчитаем выпуск, подставив все переменные в производственную функцию.

Таким образом модель построена, однако нам необходимо рассчитать неизвестные параметры.

\section{Идентификация параметров}

Первым делом, для удобства составим таблицу с параметрами и их крайними значениями.
\begin{table}[ht]
	\centering
	\caption{Неизвестные параметры модели}
	\label{tab:parameters}
	\begin{tabular}{|r|c|l|}
	\hline
	Параметр & Нижнее значение & Верхнее значение \\ \hline
	$\alpha$ &      0          &      1           \\
	$n$      &      -0.05      &      0.05        \\
	$\psi$   &      0          &      0.1         \\
	$g$      &      0          &      0.1         \\
	$A_0$    &      1          &      100         \\
	$\delta$ &      0.01       &    0.1           \\ \hline
	\end{tabular}%
\end{table}
Разъясним граничные значения приведенные в таблице \ref{tab:parameters}.

Ограничения $\alpha$ заданы из определения производственной функции Кобба-Дугласа.

Параметр роста численности работающих $n$ определен в диапозоне от $-0.05$ до $0.05$, исходя из соображений, что каждый период кто-то становится, а кто-то перестает быть трудоспособным.
Теоретически изменение численности трудящихся не может быть более чем на 5 процентов.

Константа $\psi$, показывающая увеличение эффективности квалифицированного рабочего от обучения, не может быть более чем 10 процентов за период.

Исходя из текущего уровня мировой науки, увеличение технологического прогресса $g$ задано в таких ограничениях, так как не может быть более чем на 10 процентов в год.

Уровень начального технологического прогресса $A_0$, может быть абсолютно любым, однако из логических соображений технологии не могут влиять на выпуск более чем в 100 раз.

Ежегодный износ капитала $\delta$ наблюдается всегда, однако в спокойное время\footnote{Отсутствие военных действий.} не может быть более чем 10 процентов.

\section{Рассчет параметров}

Определимся с нашей основной задачей.
Мы хотим подобрать, в соответсвии с ограничениями, такие параметры модели, что они будут наиболее близкими с статистическим данным.
Для этого введем индекс Тейла:
\begin{equation*}
	T_{X} = \sqrt{\cfrac{\sum\limits_{t = 0}^{8}\left(X(t) - X_{stat}(t)\right)^2}{\sum\limits_{t = 0}^{8}(X(t))^2 - (X_{stat}(t))^2}}\text{,}
\end{equation*}
где $X$ это сравниваемая переменная.
Если $T_{X} = 0$, значит что полученные данные совпадают со статистическими.

Для того, чтобы получить результат близости полученных и статистических данных, произведем свертку критериев Тейла по всем переменным.
Получим:
\begin{equation*}
S=\prod\limits_{X=Y,L}\left(1 -T_{X}\right)
\end{equation*}

Таким образом, требуется найти максимум свертки при заданных ограничениях:
\begin{equation*}
\max_{\substack{\alpha^- < \alpha < \alpha^+ \\ n^- < n < n^+ \\ \psi^- < \psi < \psi^+ \\ g^- < g < g^+ \\ A_0^- < A_0 < A_0^+ \\ \delta^- < \delta < \delta^+}} S
\end{equation*}

В дальнейшем, все подобные вычисления можно произвести в рассчете на душу населения.
