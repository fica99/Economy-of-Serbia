\chapter{Модель Республики Сербия}
Мы разобрались с основами модели Солоу, теперь надо ее построить.
В общем случае модель состоит из нескольких уравнений, которые описывают взаимосвязи между набором эндогенных переменных --- переменных, значения которых определяются внутри модели.
Можно заметить, что уравнения, описывающие взаимосвязи между эндогенными переменными включают в себя различные параметры и экзогенные переменные.
Параметры --- это постоянные величины.
Экзогенные переменные --- это величины, которые могут меняться во времени, однако их значения определяются вне модели.

После объяснения этих понятий мы готовы построить модель.
Решение модели означает получение значений каждой эндогенной переменной, когда даны значения для экзогенных переменных и параметров.
В идеале хотелось бы иметь возможность выражать каждую эндогенную переменную как функцию только от экзогенных переменных и параметров.

В предыдущем разделе мы ввели два ключевых уравнения модели Солоу:
\begin{align*}
	Y=K^{\alpha}(AH)^{1-\alpha}\text{,}\\
	K'=s_{K}Y - \delta K \text{.}
\end{align*}

Переопределим обозначение $t$, чтобы было удобнее работать.
Пусть 2010 год будет базисным годом, тогда $t = \text{год} - 2010$.

Для начала рассчитаем численность труда, пусть $L_0 = L_{stat}(0)$.
Теперь воспользуемся формулой:
\begin{equation*}
	L(t) = L_0 e^{nt}\text{,}
\end{equation*}
где $n$ --- темп прироста численности трудящегося населения.

Теперь можно вычислить квалифицированный труд.
Предположим, что за единицу времени обучения труд становится продуктивнее на 5 процентов, значит $\psi= 0.05$.
Также сделаем предположение, что только половина трудящихся страны является квалифицированной.
Из-за отстутствия статистики приходится делать предположения.
В зависимости от уровня образования, обучение занимает разное время.
В среднем квалифицированное обучение в Республике Сербия занимает 4 года, значит $u = 4$.
Теперь мы можем воспользоваться формулой:
\begin{equation*}
H = 0.5L + 0.5 e^{\psi u}L\text{.}
\end{equation*}


% Как было показано ранее, физический капитал накапливается путем инвестирования некоторого объема продукции, а значит можем посчитать начальный запас капитала:
% \begin{equation*}
% 	K_0= \varkappa + \sum\limits_{t = -17}^{0}J_{stat}(t)\text{,}
% \end{equation*}
% где $\varkappa$ --- некая константа для приближения результатов к статистике.
