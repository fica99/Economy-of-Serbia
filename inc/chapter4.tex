\chapter{Модель Республики Сербия}

Мы разобрались с основами модели Солоу, теперь надо ее построить.
Модель --- набор нескольких уравнений, описыващих связи между эндогенными переменными (определяются внутри модели).
Эти уравнения включают в себя различные параметры и экзогенные переменные.
Параметры --- это постоянные величины.
Экзогенные переменные --- это величины, которые могут меняться во времени, однако их значения определяются вне модели.

\section{Спецификация модели}

После объяснения этих понятий мы готовы построить модель.
Построение модели означает получение значений каждой эндогенной переменной, когда даны значения для экзогенных переменных и параметров.
В идеале хотелось бы иметь возможность выражать каждую эндогенную переменную как функцию только от экзогенных переменных и параметров.

В предыдущем разделе мы ввели два ключевых уравнения модели Солоу:
\begin{align*}
	Y=K^{\alpha}(AH)^{1-\alpha}\text{,}\\
	K'=sY - \delta K \text{.}
\end{align*}

Пусть 2010 год будет базисным годом, тогда $t = \text{год} - 2010$.

Для начала рассчитаем труд.
Пусть $L_0 = L_{stat}(0)$, теперь воспользуемся формулой:
\begin{equation*}
	L(t) = L_0 e^{nt}\text{,}
\end{equation*}
где $n$ --- темп прироста численности трудящегося населения.

Воспользовавшись количеством трудящихся, можно вычислить квалифицированный труд.
Из-за отсутствия статистики сделаем предположение, что только половина трудящихся страны является квалифицированной.
В зависимости от уровня образования, обучение занимает разное время.
В среднем квалифицированное обучение в Республике Сербия занимает 4 года, значит $u = 4$.
Модифицировав формулу подсчета квалифицированного труда в соответствии с предположениями, получим формулу:
\begin{equation*}
H(t) = 0.5L(t) + 0.5 e^{\psi u}L(t)\text{,}
\end{equation*}
где $\psi$ --- константа, показывающая увеличение эффективности труда за единицу времени обучения.

Теперь займемся технологиями.
Так как уровень технологического прогресса существенно влияет на производство, нам необходимо его определить:
\begin{equation*}
A(t) = A_{0}e^{gt}\text{,}
\end{equation*}
где $g$ --- это параметр, отвечающий за технологический рост.
Начальный уровень технологий $A_0$ --- параметр, который определяется экзогенно.

Как было показано ранее, физический капитал накапливается путем инвестирования некоторого объема продукции, а значит можно посчитать начальный запас капитала, как сумму инвестиций c учетом амортизации за прошлые года:
\begin{equation*}
	K_0= \varkappa + \sum\limits_{t = -17}^{0}I_{stat}(t) e^{\delta t}\text{,}
\end{equation*}
где $\varkappa$ --- некая константа, отвечающая за накопленный капитал, $\delta$ --- параметр износа капитала.

Теперь мы можем рассчитать капитал по формуле:
\begin{equation*}
	K'=sY - \delta K \text{,}
\end{equation*}
где $s$ --- инвестиционная постоянная.
Преобразуем это уравнение к более удобному виду для вычислений:
\begin{equation*}
	K_{t+1} - K_{t} = sY_t - \delta K_{t}\text{.}
\end{equation*}
Перенесем $K_{t}$ в правую часть уравения:
\begin{equation*}
K_{t+1} = s Y_t + (1 - \delta)K_{t}\text{.}
\end{equation*}
Теперь мы можем подставить производственную функцию и получим:
\begin{equation*}
	K_{t+1} = sK_{t}^{\alpha}(A_tH_t)^{1-\alpha} + (1 - \delta)K_{t}\text{.}
\end{equation*}
В этой формуле все переменные определены, поэтому без проблем можно рассчитать капитал.

Теперь все переменные производственной функции нам известны, посчитаем выпуск.

Из предположений о инвестициях получим, что:
\begin{equation*}
	I(t)=sY(t)=sK_{t}^{\alpha}(A_tH_t)^{1-\alpha}\text{,}
\end{equation*}
из этого следует, что потребление можно вычислить по формуле:
\begin{equation*}
	C(t) = (1-s)Y(t)= (1-s)K_{t}^{\alpha}(A_tH_t)^{1-\alpha}\text{.}
\end{equation*}

Таким образом модель построена, однако нам необходимо рассчитать неизвестные параметры.

\section{Идентификация параметров}

Первым делом, для удобства составим таблицу с параметрами и их крайними значениями.
Разъясним граничные значения приведенные в таблице \ref{tab:parameters}.
\begin{table}[ht]
	\centering
	\caption{Неизвестные параметры модели.}
	\label{tab:parameters}
	\begin{tabular}{|r|c|l|}
	\hline
	Параметр & Нижнее значение & Верхнее значение \\ \hline
	$\alpha$ &      0          &      1           \\
	$n$      &      -0.05      &      0.05        \\
	$\psi$   &      0          &      0.1         \\
	$g$      &      0          &      0.1         \\
	$A_0$    &      1          &      100         \\
	$s$      &      0          &      1           \\
	$\delta$ &      0.01       &    0.1           \\ \hline
	\end{tabular}%
\end{table}

Ограничения $\alpha$ заданы из определения производственной функции Кобба-Дугласа.

Параметр роста численности работающих $n$ определен в диапозоне от $-0.05$ до $0.05$, исходя из соображений, что каждый период кто-то становится трудоспособным, а кто-то перестает им быть.
Теоретически изменение численности трудящихся не может быть более чем на 5 процентов в год.

Увеличение эффективности квалифицированного рабочего от обучения $\psi$ не может расти более чем на 10 процентов в год.

Исходя из текущего уровня мировой науки, увеличение технологического прогресса $g$ задано в таких ограничениях, так как не может расти более чем на 10 процентов в год.

Уровень начального технологического прогресса $A_0$, может быть абсолютно любым, однако из логических соображений технологии не могут влиять на выпуск более чем в 100 раз.

Инвестиции $s$ в будущий выпуск могут присутствовать, а могут и отсутствовать.

Ежегодный износ капитала $\delta$ наблюдается всегда, однако в спокойное время\footnote{Отсутствие военных действий.} не может быть более чем на 10 процентов.

\section{Расчет параметров}

Определимся с нашей основной задачей.
Мы хотим подобрать, в соответствии с ограничениями, такие параметры модели, что полученные переменные будут наиболее близкими к статистическим данным.
Для этого введем индекс Тейла:
\begin{equation*}
	T_{X} = \sqrt{\cfrac{\sum\limits_{t = 0}^{8}\left(X(t) - X_{stat}(t)\right)^2}{\sum\limits_{t = 0}^{8}\left((X(t))^2 + (X_{stat}(t))^2\right)}}\text{,}
\end{equation*}
где $X$ --- это сравниваемая переменная.
Если $T_{X} = 0$, то полученные данные совпадают со статистическими.

Для того, чтобы получить результат близости полученных и статистических данных, произведем свертку критериев Тейла по всем переменным.
Получим:
\begin{equation*}
S=\prod\limits_{X=Y,L,C,I}\left(1 -T_{X}\right)\text{.}
\end{equation*}

Таким образом, требуется найти максимум свертки при заданных ограничениях:
\begin{equation*}
\max_{\substack{\alpha^- < \alpha < \alpha^+ \\ n^- < n < n^+ \\ \psi^- < \psi < \psi^+ \\ g^- < g < g^+ \\ A_0^- < A_0 < A_0^+ \\ \delta^- < \delta < \delta^+\\ s^- < s < s^+}} S \text{,}
\end{equation*}
где индекс $^+$ обозначает верхнюю границу, а $^-$ нижнюю соответственно.

Существует различные принципы и методы решения подобных задач.
Наиболее удобными являются различные регрессии, основанные на методе наименьших квадратов, которые в настоящее время очень актуальны в области машинного обучения и искусственного интеллекта.
Также в решении подобных задач применяются различные интерполяционные многочлены, сплайны и другие способы поиска решения.
Однако существуют и другие методы, например, часто оптимальную норму сбережения ищут с помощью принципа максимума Понтрягина.
Все эти математические методы связаны с теорией оптимизации, задачами оптимального управления и численными методами.
Эти знания становятся все более актуальными и очень востребованными на рынке труда.

\section{Результаты}

Максимизируя свертку критериев Тейла $S$ в Microsoft Excel\footnote{Microsoft Excel --- программа для работы с электронными таблицами, созданная корпорацией Microsoft.} с помощью метода Ньютона, были получены параметры, представленные в таблице \ref{tab::res_params}.
Результаты были округлены до тысячных.
Так как при данном методе решения используется конечное число итераций, то результат является приближенным.
Такой метод позволяет найти только локальные экстремумы, однако были получены достаточно близкие результаты к статистике.

\begin{table}[ht]
	\centering
	\caption{Найденные параметры модели.}
	\label{tab::res_params}
	\begin{tabular}{|r|l|}
	\hline
	Параметр & Значение         \\ \hline
	$\alpha$ &      0.944       \\
	$n$      &      0.005       \\
	$\psi$   &      0.1         \\
	$g$      &      0.1         \\
	$A_0$    &      50          \\
	$s$      &      0.172       \\
	$\delta$ &      0.096       \\ \hline
	\end{tabular}%
\end{table}

Значения критериев Тейла представлены в таблице \ref{tab::res_crit_teil}.
Данные результаты дают свертку $S=0.864$.
Таблица смоделированных данных представлена в приложении \ref{cha:second_app}.

\begin{table}[ht]
	\centering
	\caption{Найденные критерии Тейла.}
	\label{tab::res_crit_teil}
	\begin{tabular}{|r|l|}
	\hline
	Критерий Тейла & Значение         \\ \hline
	$T_{Y}$        &      0.042       \\
	$T_{C}$        &      0.034       \\
	$T_{I}$        &      0.06        \\
	$T_{L}$        &      0.006       \\ \hline
	\end{tabular}%
\end{table}

Начальный уровень технологического прогресса $A_0$ взят произвольно.
Произвольное изменение значений этого показателя показало, что начальный уровень технологического прогресса не столь существенно меняет результат.
Существенную роль играет ежегодный рост технологического прогресса.

Построим графики сравнения статистических и полученных результатов.

Минимальный критерий Тейла является критерий по количеству трудящегося населения $L$, а значит, что этот показатель наиболее близок к статистическим данным.

\begin{center}
	\label{f::l}
	\captionof{figure}{Сравнение статистических и полученных результатов по количеству трудящегося населения.}
	\begin{tikzpicture}
		\begin{axis}[
			xlabel = $t$,
			ylabel = $L(t)$,
			table/col sep = semicolon,
			height = 0.3\paperheight,
			width = 0.5\paperwidth,
			xmin = 0,
			xmax = 8,
			/pgf/number format/1000 sep={},
			legend pos = south east,
		]
		\legend{$L$, $L_{stat}$};
		\addplot[solid,draw=blue] table[x={Year}, y={L}] {tables/L.csv};
		\addplot[dashed,draw=red] table[x={Year}, y={Lstat}] {tables/L.csv};
		\end{axis}
	\end{tikzpicture}
\end{center}

Наиболее значимым макроэкономическим показателем является ВВП $Y$.

\begin{center}
	\captionof{figure}{Сравнение статистических и полученных результатов по ВВП.}
	\begin{tikzpicture}
		\begin{axis}[
			xlabel = $t$,
			ylabel = $Y(t)$,
			table/col sep = semicolon,
			height = 0.3\paperheight,
			width = 0.5\paperwidth,
			xmin = 0,
			xmax = 8,
			/pgf/number format/1000 sep={},
			legend pos = south east,
		]
		\legend{$Y$, $Y_{stat}$};
		\addplot[solid,draw=blue] table[x={Year}, y={Y}] {tables/Y.csv};
		\addplot[dashed,draw=red] table[x={Year}, y={Ystat}] {tables/Y.csv};
		\end{axis}
	\end{tikzpicture}
\end{center}

Продемонстрируем динамику статистического и расчетного потребления $C$.

\begin{center}
	\captionof{figure}{Сравнение статистических и полученных результатов по потреблению .}
	\begin{tikzpicture}
		\begin{axis}[
			xlabel = $t$,
			ylabel = $C(t)$,
			table/col sep = semicolon,
			height = 0.3\paperheight,
			width = 0.5\paperwidth,
			xmin = 0,
			xmax = 8,
			/pgf/number format/1000 sep={},
			legend pos = south east,
		]
		\legend{$C$, $C_{stat}$};
		\addplot[solid,draw=blue] table[x={Year}, y={C}] {tables/C.csv};
		\addplot[dashed,draw=red] table[x={Year}, y={Cstat}] {tables/C.csv};
		\end{axis}
	\end{tikzpicture}
\end{center}

Построив графики потребления, можно построить график сравнения инвестиций $I$.

\begin{center}
	\captionof{figure}{Сравнение статистических и полученных результатов по инвестициям.}
	\begin{tikzpicture}
		\begin{axis}[
			xlabel = $t$,
			ylabel = $I(t)$,
			table/col sep = semicolon,
			height = 0.3\paperheight,
			width = 0.5\paperwidth,
			xmin = 0,
			xmax = 8,
			/pgf/number format/1000 sep={},
			legend pos = south east,
		]
		\legend{$I$, $I_{stat}$};
		\addplot[solid,draw=blue] table[x={Year}, y={I}] {tables/I.csv};
		\addplot[dashed,draw=red] table[x={Year}, y={Istat}] {tables/I.csv};
		\end{axis}
	\end{tikzpicture}
\end{center}

Таким образом, построив графики, видно, что полученные данные достаточно близки к статистике.
Результат данного построения можно считать удовлетворительным.

Манипулирая значением параметров, можно увидеть динамику изменения результатов.
Динамика способна ответить на вопрос в каком направлении стоит развиваться для выхода на траекторию сбалансированного роста.
Полученная модель способна предсказать приблизительные макроэкономические показатели.

Опираясь на приведенные выше графики, можно заметить , что все данные ежегодно растут.
Из вышесказанного можно сделать вывод, что Республика Сербия после тяжелого периода в своей истории уверенно растет и становится все более благоприятной страной для жизни.

Существует множество математических моделей, которые используются не только в макроэкономике.
Этот инструмент являются универсальным для многих задач и в настоящее время очень актуален в мире.

