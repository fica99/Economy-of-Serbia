%%% Работа с русским языком
\usepackage[english,russian]{babel}   %% загружает пакет многоязыковой вёрстки
\usepackage{fontspec}      %% подготавливает загрузку шрифтов Open Type, True Type и др.
\defaultfontfeatures{Ligatures={TeX},Renderer=Basic}  %% свойства шрифтов по умолчанию
\setmainfont[Ligatures={TeX,Historic}]{Times New Roman} %% задаёт основной шрифт документа
\setsansfont{Comic Sans MS}                    %% задаёт шрифт без засечек
\setmonofont{Courier New}
\usepackage{indentfirst}
\frenchspacing

%Автоматизировать создание перечня условных обозначений и сокращений
\usepackage[notintoc]{nomencl}
\makenomenclature

% Пакет для получения имени текущей секции
\usepackage{nameref}
\makeatletter


% Пакет для точек в оглавлении
\usepackage{tocloft}

%%% Страница
\usepackage{geometry} % Простой способ задавать поля
	\geometry{top=20mm}
	\geometry{bottom=20mm}
	\geometry{left=30mm}
	\geometry{right=20mm}

\usepackage{setspace} % Интерлиньяж
\onehalfspacing % Интерлиньяж 1.5
%\doublespacing % Интерлиньяж 2
%\singlespacing % Интерлиньяж 1

% Размер абзацного отступа – 5 знаков (1,25 см).
\setlength{\parskip}{1.25cm}

%%% Дополнительная работа с математикой
\usepackage{amsmath,amsfonts,amssymb,amsthm,mathtools} % AMS
\usepackage{icomma} % "Умная" запятая: $0,2$ --- число, $0, 2$ --- перечисление

 %% Номера формул
\mathtoolsset{showonlyrefs=true} % Показывать номера только у тех формул, на которые есть \eqref{} в тексте.
%\usepackage{leqno} % Нумерация формул слева


%%% Работа с картинками
\usepackage{graphicx}  % Для вставки рисунков
\graphicspath{{images/}{images2/}}  % папки с картинками
\setlength\fboxsep{3pt} % Отступ рамки \fbox{} от рисунка
\setlength\fboxrule{1pt} % Толщина линий рамки \fbox{}
\usepackage{wrapfig} % Обтекание рисунков текстом

%%% Работа с таблицами
\usepackage{array,tabularx,tabulary,booktabs} % Дополнительная работа с таблицами
\usepackage{longtable}  % Длинные таблицы
\usepackage{multirow} % Слияние строк в таблице

% Все перекрестные ссылки становятся гиперссылками
\usepackage{hyperref}
\usepackage[usenames,dvipsnames,svgnames,table,rgb]{xcolor}
\hypersetup{				% Гиперссылки
	unicode=true,           % русские буквы в раздела PDF
	pdftitle={Построение математической модели экономики Сербии},   % Заголовок
	pdfauthor={Црнобрня Филипп},      % Автор
	pdfsubject={Математическая модель экономики Сербии},      % Тема
	pdfcreator={Црнобрня Филипп}, % Создатель
	pdfproducer={Црнобрня Филипп}, % Производитель
	pdfkeywords={модель} {экономика Сербии} {математика}, % Ключевые слова
	colorlinks=true,       	% false: ссылки в рамках; true: цветные ссылки
	linkcolor=black,          % внутренние ссылки
	citecolor=black,        % на библиографию
	filecolor=black,      % на файлы
	urlcolor=black           % на URL
}

\usepackage{csquotes} % Еще инструменты для ссылок

% Пакет для работы с библиографией
\usepackage[backend=biber,bibencoding=utf8,sorting=nyvt,maxcitenames=2,style=numeric]{biblatex}
\addbibresource{base/base.bib}
