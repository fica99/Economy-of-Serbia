\renewcommand{\epsilon}{\ensuremath{\varepsilon}}
\renewcommand{\phi}{\ensuremath{\varphi}}
\renewcommand{\kappa}{\ensuremath{\varkappa}}
\renewcommand{\le}{\ensuremath{\leqslant}}
\renewcommand{\leq}{\ensuremath{\leqslant}}
\renewcommand{\ge}{\ensuremath{\geqslant}}
\renewcommand{\geq}{\ensuremath{\geqslant}}
\renewcommand{\emptyset}{\varnothing}

% Имя текущей секции
\newcommand*{\currentname}{\@currentlabelname}

% Команда для начала Введения
\newcommand{\intro}{%
\pagebreak
\section*{Введение}
\addcontentsline{toc}{section}{\currentname}
}

%% Перенос знаков в формулах (по Львовскому)
\newcommand*{\hm}[1]{#1\nobreak\discretionary{}
{\hbox{$\mathsurround=0pt #1$}}{}}

% Точки в оглавлении
\renewcommand{\cftsecleader}{\cftdotfill{\cftdotsep}}

% Изменение заголовка перечня
\renewcommand{\nomname}{Перечень сокращений и условных обозначений}

% Позволяет одновременно печатать условное обозначение в тексте документа и добавлять его в перечень
\newcommand*{\nom}[2]{#1\nomenclature{#1}{#2}}

%%% Теоремы
\theoremstyle{plain} % Это стиль по умолчанию, его можно не переопределять.
\newtheorem{theorem}{Теорема}[section]
\newtheorem{proposition}[theorem]{Утверждение}

\theoremstyle{definition} % "Определение"
\newtheorem{corollary}{Следствие}[theorem]
\newtheorem{problem}{Задача}[section]

\theoremstyle{remark} % "Примечание"
\newtheorem*{nonum}{Решение}

% Команда для начала новой главы
\newcommand{\glava}[1]{%
\pagebreak
\setcounter{subsection}{0}
\addtocounter{section}{1}
\section*{Глава \arabic{section}. #1}
\addcontentsline{toc}{section}{\currentname}
}

% Команда для начала нового приложения
\newcounter{appcount}
\renewcommand{\appendix}{%
\pagebreak
\refstepcounter{appcount}
\section*{Приложение \arabic{appcount}}
\addcontentsline{toc}{section}{\currentname}
}
