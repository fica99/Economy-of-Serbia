\documentclass[a4paper, 14pt]{extarticle}



%%% Работа с русским языком
\usepackage[english,russian]{babel}   %% загружает пакет многоязыковой вёрстки
\usepackage{fontspec}      %% подготавливает загрузку шрифтов Open Type, True Type и др.
\defaultfontfeatures{Ligatures={TeX},Renderer=Basic}  %% свойства шрифтов по умолчанию
\setmainfont[Ligatures={TeX,Historic}]{Times New Roman} %% задаёт основной шрифт документа
\setsansfont{Comic Sans MS}                    %% задаёт шрифт без засечек
\setmonofont{Courier New}
\usepackage{indentfirst}
\frenchspacing

%Автоматизировать создание перечня условных обозначений и сокращений
\usepackage[intoc]{nomencl}
\makenomenclature
% Пакет для точек в оглавлении
\usepackage{tocloft}

% Пакет для обрывания страницы перед новым section
\usepackage{titlesec}

%%% Страница
\usepackage{geometry} % Простой способ задавать поля
	\geometry{top=20mm}
	\geometry{bottom=20mm}
	\geometry{left=30mm}
	\geometry{right=20mm}

\usepackage{setspace} % Интерлиньяж
\onehalfspacing % Интерлиньяж 1.5
%\doublespacing % Интерлиньяж 2
%\singlespacing % Интерлиньяж 1

% Размер абзацного отступа – 5 знаков (1,25 см).
\setlength{\parskip}{1.25cm}

%%% Дополнительная работа с математикой
\usepackage{amsmath,amsfonts,amssymb,amsthm,mathtools} % AMS
\usepackage{icomma} % "Умная" запятая: $0,2$ --- число, $0, 2$ --- перечисление

 %% Номера формул
\mathtoolsset{showonlyrefs=true} % Показывать номера только у тех формул, на которые есть \eqref{} в тексте.
%\usepackage{leqno} % Нумерация формул слева


%%% Работа с картинками
\usepackage{graphicx}  % Для вставки рисунков
\graphicspath{{images/}{images2/}}  % папки с картинками
\setlength\fboxsep{3pt} % Отступ рамки \fbox{} от рисунка
\setlength\fboxrule{1pt} % Толщина линий рамки \fbox{}
\usepackage{wrapfig} % Обтекание рисунков текстом

%%% Работа с таблицами
\usepackage{array,tabularx,tabulary,booktabs} % Дополнительная работа с таблицами
\usepackage{longtable}  % Длинные таблицы
\usepackage{multirow} % Слияние строк в таблице



\renewcommand{\epsilon}{\ensuremath{\varepsilon}}
\renewcommand{\phi}{\ensuremath{\varphi}}
\renewcommand{\kappa}{\ensuremath{\varkappa}}
\renewcommand{\le}{\ensuremath{\leqslant}}
\renewcommand{\leq}{\ensuremath{\leqslant}}
\renewcommand{\ge}{\ensuremath{\geqslant}}
\renewcommand{\geq}{\ensuremath{\geqslant}}
\renewcommand{\emptyset}{\varnothing}

%% Введение
\newcommand{\intro}{\section{Введение}}

%% Новая страница для каждого section
\newcommand{\sectionbreak}{\clearpage}

%% Перенос знаков в формулах (по Львовскому)
\newcommand*{\hm}[1]{#1\nobreak\discretionary{}
{\hbox{$\mathsurround=0pt #1$}}{}}

% Точки в оглавлении
\renewcommand{\cftsecleader}{\cftdotfill{\cftdotsep}}

% Изменение заголовка перечня
\renewcommand{\nomname}{Перечень сокращений и условных обозначений}

% Позволяет одновременно печатать условное обозначение в тексте документа и добавлять его в перечень
\newcommand*{\nom}[2]{#1\nomenclature{#1}{#2}}

%%% Теоремы
\theoremstyle{plain} % Это стиль по умолчанию, его можно не переопределять.
\newtheorem{theorem}{Теорема}[section]
\newtheorem{proposition}[theorem]{Утверждение}

\theoremstyle{definition} % "Определение"
\newtheorem{corollary}{Следствие}[theorem]
\newtheorem{problem}{Задача}[section]

\theoremstyle{remark} % "Примечание"
\newtheorem*{nonum}{Решение}


\begin{document}

\tableofcontents


\printnomenclature[2cm]
\phantomsection
\addcontentsline{toc}{section}{\nomname}


Целью работы является создание всякой всячины. Для достижения поставленной цели необходимо решить следующие задачи:

\begin{itemize}
\item проанализировать существующую всячину;
\item спроектировать свою, новую всячину;
\item изготовить всякую всячину;
\item проверить её работоспособность.
\end{itemize}

Вот так-то. А этот абзац вставлен для визуальной оценки отступа от перечня до следующего абзаца.

Целью работы является создание всякой всячины. Для достижения поставленной цели необходимо решить следующие задачи:

\begin{itemize}
\item проанализировать существующую всячину;
\item спроектировать свою, новую всячину;
\item изготовить всякую всячину;
\item проверить её работоспособность.
\end{itemize}

Вот так-то. А этот абзац вставлен для визуальной оценки отступа от перечня до следующего абзаца.


Целью работы является создание всякой всячины. Для достижения поставленной цели необходимо решить следующие задачи:

\begin{itemize}
\item проанализировать существующую всячину;
\item спроектировать свою, новую всячину;
\item изготовить всякую всячину;
\item проверить её работоспособность.
\end{itemize}

Вот так-то. А этот абзац вставлен для визуальной оценки отступа от перечня до следующего абзаца.


\Introduction

Республика Сербия --- индустриально-аграрная страна, расположенная на юго-востоке Европы (Западные Балканы).
Республика является связывающим звеном между Центральной Европой и Ближним Востоком.
Через страну проходят торговые и транспортные пути.
Республика Сербия имеет запасы полезных ископаемых --- руды цветных металлов, каменный уголь и т.~д.
Через страну протекают крупные реки (Дунай, Сава, Дрина и т.~д.).

На рубеже 1980 -- 1990 годов страна (на тот момент Югославия) была экономически развитой.
Однако политические события 90-х (уничтожение промышленности в ходе многочисленных воздушных атак НАТО, война, санкции ООН, утрата всех связей внутри бывшей Югославии и т.~д.) оказали негативное влияние на экономическое и политическое положение страны.

За последние годы в Сербии начался стремительный экономический рост, возобновились иностранная финансовая помощь и инвестиции.
За 10-летний период по данным МВФ сербская экономика выросла почти на 20 процентов.

Сельское хозяйство, промышленность и сектор услуг являются основными источниками доходов Сербии.
Они внесли большой вклад в динамику роста ВВП.
Основная отрасль сельского хозяйства --- растениеводство.
В обрабатывающей промышленности ведущее место занимают машиностроение и металлообработка.
Также уверенными темпами развиваются ИТ и туризм.
Например, экспорт ИТ сектора в 2019 году был выше, чем экспорт доминирующего сельского хозяйства.
Примечательно, что в 2019 году Балканская республика наряду с Ирландией показала самый высокий экономический рост среди всех остальных государств Европы.
В 2020 году Сербия также может показать опережающию темпы роста экономики в Европе.

\textbf{Актуальность выбранной темы выпускной квалификационной работы } обусловлена тем, что экономика Сербии в настоящий момент находится на этапе активного развития.
Математические модели способны описать текущую экономическую ситуацию в стране и спрогнозировать как и положительные, так и отрицательные сюжеты развития государства.

Экономика государства очень сильно зависит от политической ситуации, она настолько динамична, что построив математическую модель вчера, сегодня она уже может оказаться не актуальной. В связи с вспышкой пандемии COVID-19\footnote{COVID-19 (аббревиатура от англ. COronaVIrus Disease 2019), коронавирусная инфекция 2019-nCoV --- потенциально тяжёлая острая респираторная инфекция, вызываемая коронавирусом \cite{wiki:Coronavirus_disease_2019}.}, многие страны уже оказались в неприятной экономической ситуации.
Это еще одна причина выбора темы.
Кроме того, построение математических моделей экономики государства на примере Сербии поможет разобраться как в особенностях государства, так и в математических инструментах.

\textbf{Объект исследования} --- математические модели экономического роста.

\textbf{Предмет исследования} --- математическая модель экономики, построенная на примере Республики Сербия.

\textbf{Цель данной работы} --- смоделировать динамику экономики Республики Сербия в зависимости от поведения внутренних и внешних переменных и сделать выводы.

Для реализации поставленной цели необходимо решить следующие задачи:
\begin{enumerate}
	\item Изучить теорию построения математических моделей экономики.
	\item Построить модель на примере макроэкономических данных Республики Сербия.
	\item Сделать выводы.
\end{enumerate}

Выпускная квалификационная работа состоит из содержания, перечня сокращений, введения, пяти глав, заключения, списка используемых источников и приложения.

В первой главе определяются основные термины, описываются этапы построения математических моделей, приводятся различные типы математических моделей.

Во второй главе описывается теоретическая состовляющая математической модели экономического роста Солоу.

В третьей главе рассчитываются основные макроэкономические переменные экономики Республики Сербия.

В четвертой главе проводится анализ, вычисления и построение математической модели экономики Республики Сербия, разбираются полученные результаты.

В пятой главе исследуются главные сферы экономической деятельности Сербии, оцениваются прогнозы авторитетных рейтинговых агенств (МВФ, Всемирный банк и т.~д.).




Целью работы является создание всякой всячины. Для достижения поставленной цели необходимо решить следующие задачи:

\begin{itemize}
\item проанализировать существующую всячину;
\item спроектировать свою, новую всячину;
\item изготовить всякую всячину;
\item проверить её работоспособность.
\end{itemize}

Вот так-то. А этот абзац вставлен для визуальной оценки отступа от перечня до следующего абзаца.

\end{document}
