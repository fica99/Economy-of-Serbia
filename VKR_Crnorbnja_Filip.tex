\documentclass[a4paper, 14pt]{extarticle}

%%% Работа с русским языком
\usepackage[english,russian]{babel}   %% загружает пакет многоязыковой вёрстки
\usepackage{fontspec}      %% подготавливает загрузку шрифтов Open Type, True Type и др.
\defaultfontfeatures{Ligatures={TeX},Renderer=Basic}  %% свойства шрифтов по умолчанию
\setmainfont[Ligatures={TeX,Historic}]{Times New Roman} %% задаёт основной шрифт документа
\setsansfont{Comic Sans MS}                    %% задаёт шрифт без засечек
\setmonofont{Courier New}
\usepackage{indentfirst}
\frenchspacing

%Автоматизировать создание перечня условных обозначений и сокращений
\usepackage[notintoc]{nomencl}
\makenomenclature

% Пакет для получения имени текущей секции
\usepackage{nameref}
\makeatletter


% Пакет для точек в оглавлении
\usepackage{tocloft}

%%% Страница
\usepackage{geometry} % Простой способ задавать поля
	\geometry{top=20mm}
	\geometry{bottom=20mm}
	\geometry{left=30mm}
	\geometry{right=20mm}

\usepackage{setspace} % Интерлиньяж
\onehalfspacing % Интерлиньяж 1.5
%\doublespacing % Интерлиньяж 2
%\singlespacing % Интерлиньяж 1

% Размер абзацного отступа – 5 знаков (1,25 см).
\setlength{\parskip}{1cm}

%%% Дополнительная работа с математикой
\usepackage{amsmath,amsfonts,amssymb,amsthm,mathtools} % AMS
\usepackage{icomma} % "Умная" запятая: $0,2$ --- число, $0, 2$ --- перечисление

 %% Номера формул
\mathtoolsset{showonlyrefs=true} % Показывать номера только у тех формул, на которые есть \eqref{} в тексте.
%\usepackage{leqno} % Нумерация формул слева


%%% Работа с картинками
\usepackage{graphicx}  % Для вставки рисунков
\graphicspath{{images/}{images2/}}  % папки с картинками
\setlength\fboxsep{3pt} % Отступ рамки \fbox{} от рисунка
\setlength\fboxrule{1pt} % Толщина линий рамки \fbox{}
\usepackage{wrapfig} % Обтекание рисунков текстом

%%% Работа с таблицами
\usepackage{array,tabularx,tabulary,booktabs} % Дополнительная работа с таблицами
\usepackage{longtable}  % Длинные таблицы
\usepackage{multirow} % Слияние строк в таблице

% Все перекрестные ссылки становятся гиперссылками
\usepackage{hyperref}
\usepackage[usenames,dvipsnames,svgnames,table,rgb]{xcolor}
\hypersetup{				% Гиперссылки
	unicode=true,           % русские буквы в раздела PDF
	pdftitle={Построение математической модели экономики Сербии},   % Заголовок
	pdfauthor={Црнобрня Филипп},      % Автор
	pdfsubject={Математическая модель экономики Сербии},      % Тема
	pdfcreator={Црнобрня Филипп}, % Создатель
	pdfproducer={Црнобрня Филипп}, % Производитель
	pdfkeywords={модель} {экономика Сербии} {математика}, % Ключевые слова
	colorlinks=true,       	% false: ссылки в рамках; true: цветные ссылки
	linkcolor=black,          % внутренние ссылки
	citecolor=black,        % на библиографию
	filecolor=black,      % на файлы
	urlcolor=black           % на URL
}

\usepackage{csquotes} % Еще инструменты для ссылок

% Пакет для работы с библиографией
\usepackage[backend=biber,bibencoding=utf8,sorting=nyvt,maxcitenames=2,style=numeric]{biblatex}
\addbibresource{base/base.bib}


\renewcommand{\epsilon}{\ensuremath{\varepsilon}}
\renewcommand{\phi}{\ensuremath{\varphi}}
\renewcommand{\kappa}{\ensuremath{\varkappa}}
\renewcommand{\le}{\ensuremath{\leqslant}}
\renewcommand{\leq}{\ensuremath{\leqslant}}
\renewcommand{\ge}{\ensuremath{\geqslant}}
\renewcommand{\geq}{\ensuremath{\geqslant}}
\renewcommand{\emptyset}{\varnothing}

%% Введение
\newcommand{\intro}{\section{Введение}}

%% Новая страница для каждого section
\newcommand{\sectionbreak}{\clearpage}

%% Перенос знаков в формулах (по Львовскому)
\newcommand*{\hm}[1]{#1\nobreak\discretionary{}
{\hbox{$\mathsurround=0pt #1$}}{}}

% Точки в оглавлении
\renewcommand{\cftsecleader}{\cftdotfill{\cftdotsep}}

% Изменение заголовка перечня
\renewcommand{\nomname}{Перечень сокращений и условных обозначений}

% Позволяет одновременно печатать условное обозначение в тексте документа и добавлять его в перечень
\newcommand*{\nom}[2]{#1\nomenclature{#1}{#2}}

%%% Теоремы
\theoremstyle{plain} % Это стиль по умолчанию, его можно не переопределять.
\newtheorem{theorem}{Теорема}[section]
\newtheorem{proposition}[theorem]{Утверждение}

\theoremstyle{definition} % "Определение"
\newtheorem{corollary}{Следствие}[theorem]
\newtheorem{problem}{Задача}[section]

\theoremstyle{remark} % "Примечание"
\newtheorem*{nonum}{Решение}


\begin{document}

\input{inc/contents.tex}

\pagebreak
\printnomenclature[2cm]
\addcontentsline{toc}{section}{\nomname}


Целью работы является создание всякой всячины. Для достижения поставленной цели необходимо решить следующие задачи:

\begin{itemize}
\item проанализировать существующую всячину;
\item спроектировать свою, новую всячину;
\item изготовить всякую всячину;
\item проверить её работоспособность.
\end{itemize}

Вот так-то. А этот абзац вставлен для визуальной оценки отступа от перечня до следующего абзаца.

Целью работы является создание всякой всячины. Для достижения поставленной цели необходимо решить следующие задачи:

\begin{itemize}
\item проанализировать существующую всячину;
\item спроектировать свою, новую всячину;
\item изготовить всякую всячину;
\item проверить её работоспособность.
\end{itemize}

Вот так-то. А этот абзац вставлен для визуальной оценки отступа от перечня до следующего абзаца.


Целью работы является создание всякой всячины. Для достижения поставленной цели необходимо решить следующие задачи:

\begin{itemize}
\item проанализировать существующую всячину;
\item спроектировать свою, новую всячину;
\item изготовить всякую всячину;
\item проверить её работоспособность.
\end{itemize}

Вот так-то. А этот абзац вставлен для визуальной оценки отступа от перечня до следующего абзаца.


\pagebreak
\intro
\addcontentsline{toc}{section}{\currentname}

Сербия --- индустриально-аграрная страна, расположенная на юго-востоке Европы (центральной части Балканского полуострова).
Через республику проходят важные транспортные и торговые пути, соединяющие Западную и Центральную Европу с Ближним и Средним Востоком.
Сербия располагает значительными сырьевыми ресурсами --- запасами медной, свинцово-цинковой, железной, хромовой, марганцевой руды, а также каменного угля.
В стране имеются значительные гидроэнергетические ресурсы республики (реки Дунай, Морава, Дрина).

На рубеже 1980--1990 годов страна (на тот момент Югославия) являлась экономически развитой.
Однако политические события 90-х (санкции \nom{ООН}{Организация Объединённых Наций}, война, разрушение инфраструктуры и промышленности в ходе многочисленных воздушных атак \nom{НАТО}{Организация Североатлантического договора, Североатлантический Альянс (англ. North Atlantic Treaty Organization)}, утрата торговых связей внутри бывшей Югославии и т.~д.) оказали негативное влияние на экономическое и политическое положение страны.

Однако за последние годы в Сербии начался стремительный экономический рост, возобновились иностранная финансовая помощь и инвестиции.
За 10-летний период по данным \nom{МВФ}{Международный Валютный Фонд} сербская экономика выросла почти на 20\%.

Сельское хозяйство, промышленность и сектор услуг являются основными источниками доходов Сербии.
Они внесли большой вклад в динамику роста \nom{ВВП}{Валовой внутренний продукт}.
Основная отрасль сельского хозяйства --- растениеводство.
В обрабатывающей промышленности ведущее место занимают машиностроение и металлообработка.
Также уверенными темпами развиваются \nom{ИТ}{Информационные технологии} и туризм.
Например, экспорт ИТ сектора в 2019 году был выше, чем экспорт доминирующего сельского хозяйства.
Примечательно, что в 2019 году Балканская республика наряду с Ирландией показала самый высокий экономический рост среди всех остальных государств Европы.
В 2020 году Сербия также может показать опережающию темпы роста экономики в Европе.

\underline{Актуальность выбранной темы выпускной квалификационной работы } обусловлена тем, что экономика Сербии в настоящий момент находится на этапе активного развития.
Математические модели способны описать текущую экономическую ситуацию в стране и спрогнозировать как и положительные, так и отрицательные сюжеты развития государства.

Экономика государства очень сильно зависит от политической ситуации, она настолько динамична, что построив математическую модель вчера, сегодня она уже может оказаться не актуальной. В связи с пандемией короновируса, многие страны уже оказались в неприятной экономической ситуации.
Это еще одна причина выбора темы.
Кроме того, построение математических моделей экономики государства на примере Сербии поможет разобраться как в особенностях государства, так и в математических инструментах.

\underline{Объект исследования} --- математические модели экономического роста.

\underline{Предмет исследования} --- математические модели эк, построенные на примере Республики Сербии.

\underline{Цель данной работы} --- создать инструмент прогнозирования динамики экономики Республики Сербии в зависимости от поведения внутренних и внешних переменных, сделать выводы и построить прогнозы по полученным данным.

В связи с поставленной целью необходимо решить следующие задачи:
\begin{enumerate}
	\item Изучить особенности экономики Сербии.
	\item Определить основные факторы влияющие на экономический рост.
	\item Изучить теорию построения математических моделей экономики.
	\item Построить модели на примере данных Республики Сербии.
	\item Сделать выводы и прогнозы, основанные на моделях.
\end{enumerate}

Выпускная квалификационная работа состоит из содержания, перечня сокращений, введения, четырех глав, заключения, списка используемых источников и приложений.

В первой главе исследуются экономическая и политическая ситуация Сербии. Оцениваются главные сферы экономической деятельности, описываются прогнозы авторитетных рейтинговых агенств (МВФ и т.~д.).

Во второй главе приводятся найденные данные, и рассчитываются основные переменные.

В четвертой главе проводится анализ, вычисления и построение математических моделей экономики Республики Сербии. В этой же главе приводятся как положительные, так и негативные прогнозы развития государственной экономики.





Целью работы является создание всякой всячины. Для достижения поставленной цели необходимо решить следующие задачи:

\begin{itemize}
\item проанализировать существующую всячину;
\item спроектировать свою, новую всячину;
\item изготовить всякую всячину;
\item проверить её работоспособность.
\end{itemize}

Вот так-то. А этот абзац вставлен для визуальной оценки отступа от перечня до следующего абзаца.

\end{document}
